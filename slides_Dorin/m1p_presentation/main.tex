%\documentclass[c]{beamer}  % [t], [c], или [b] --- вертикальное выравнивание на слайдах (верх, центр, низ)
%\documentclass[handout]{beamer} % Раздаточный материал (на слайдах всё сразу)


\documentclass[9pt,pdf]{beamer} % Соотношение сторон
\usepackage[labelfont=bf]{caption}
\usepackage{subfig} % for subfigures
\usepackage{bm}
%\usetheme{Bergen}
\usetheme{Berlin}
%\setbeamertemplate{footline}[frame number]
%\usetheme{Warsaw}

%\useoutertheme{тема}
%\useinnertheme{тема} 

\setbeamercolor{background canvas}{bg=white}
\useinnertheme[shadow]{rounded}

%%% Работа с русским языком
\usepackage{cmap}					% поиск в PDF
\usepackage{mathtext} 				% русские буквы в формулах
\usepackage[T2A]{fontenc}			% кодировка
\usepackage[utf8]{inputenc}			% кодировка исходного текста
\usepackage[english,russian]{babel}	% локализация и переносы
\usepackage{graphicx}
\usepackage{geometry}
\usepackage{lipsum}
%% Beamer по-русски
\newtheorem{rtheorem}{Теорема}
\newtheorem{rproof}{Доказательство}
\newtheorem{rexample}{Пример}

%%% Дополнительная работа с математикой
\usepackage{amsmath,amsfonts,amssymb,amsthm,mathtools} % AMS
\usepackage{icomma} % "Умная" запятая: $0,2$ --- число, $0, 2$ --- перечисление

%% Номера формул
%\mathtoolsset{showonlyrefs=true} % Показывать номера только у тех формул, на которые есть \eqref{} в тексте.
%\usepackage{leqno} % Нумерация формул слева
\usepackage{bm}
%% Свои команды
\DeclareMathOperator{\sgn}{\mathop{sgn}}
\newcommand{\myfigref}[2]{~\ref{#1}.\subref{#2}}% <---- a new macro for referring to a subfigure
% \myfigref{label: fig}{label: subfig}
\newcommand{\T}{^{\mathsf{T}}}
\DeclareMathOperator*{\argmax}{arg\,max}  % in your preamble
\DeclareMathOperator*{\argmin}{arg\,min}  % in your preamble 

%%% Работа с картинками
\usepackage{graphicx}  % Для вставки рисунков
\setlength\fboxsep{3pt} % Отступ рамки \fbox{} от рисунка
\setlength\fboxrule{1pt} % Толщина линий рамки \fbox{}
\usepackage{wrapfig} % Обтекание рисунков текстом

%%% Работа с таблицами
\usepackage{array,tabularx,tabulary,booktabs} % Дополнительная работа с таблицами
\usepackage{longtable}  % Длинные таблицы
\usepackage{multirow} % Слияние строк в таблице
\usepackage[labelfont=bf]{caption}
%%% Программирование
\usepackage{etoolbox} % логические операторы

%%% Другие пакеты
\usepackage{lastpage} % Узнать, сколько всего страниц в документе.
\usepackage{soul} % Модификаторы начертания
\usepackage{csquotes} % Еще инструменты для ссылок
%\usepackage[style=authoryear,maxcitenames=2,backend=biber,sorting=nty]{biblatex}
\usepackage{multicol} % Несколько колонок

%%% Картинки
\usepackage{tikz} % Работа с графикой
\usepackage{pgfplots}
\usepackage{pgfplotstable}

\title{Предсказание показания фМРТ по видео, показанному человеку}
\author{Дорин Даниил Дмитриевич}
\date{\today}
\institute[Московский физико-технический институт]{Московский физико-технический институт }

\setbeamercovered{transparent = 15}

\begin{document}

	\begin{frame}{}
		\maketitle
	\end{frame}
%%%%%%%%%%%%%%%%%%%%%%%%%%%%%%%%%%%%%%%%%%%%%%%%%%%%%%%%%%%%%%%%%%%%%%%%%%%%%%%%%%%%%%%%%%%%%%%%%%%%%%%%%%%%%%%%%%%%%%%%%%%%%%%%%%%%%%%%%%%%%%

%%%%%%%%%%%%%%%%%%%%%%%%%%%%%%%%%%%%%%%%%%%%%%%%%%%%%%%%%%%%%%%%%%%%%%%%%%%%%%%%%%%%%%%%%%%%%%%%%%%%%%%%%%%%%%%%%%%%%%%%%%%%%%%%%%%%%%%%%%%%%%
\begin{frame}{Цель работы}
    \begin{block}{Исследуется}
    Зависимость между показаниями датчиков фМРТ и видеорядом, просматриваемым человеком.
    \end{block}
    
    \begin{block}{Требуется}
    Предложить метод аппроксимации снимка фМРТ по видеоряду и нескольким дополнительным измерениям фМРТ того же испытуемого.
    \end{block}
    \begin{block}{Основные предположения}
    \begin{itemize}
        \item Наличие корреляции между снимками и изображениями из видеоряда.
        \item Реакция мозга, фиксируемая фМРТ, на информацию, поступающую от зрительных органов, происходит не мгновенно, а с некоторой задержкой $\Delta t$.
    \end{itemize}
    \end{block}
\end{frame}
%%%%%%%%%%%%%%%%%%%%%%%%%%%%%%%%%%%%%%%%%%%%%%%%%%%%%%%%%%%%%%%%%%%%%%%%%%%%%%%%%%%%%%%%%%%%%%%%%%%%%%%%%%%%%%%%%%%%%%%%%%%%%%%%%%%%%%%%%%%%%%

%%%%%%%%%%%%%%%%%%%%%%%%%%%%%%%%%%%%%%%%%%%%%%%%%%%%%%%%%%%%%%%%%%%%%%%%%%%%%%%%%%%%%%%%%%%%%%%%%%%%%%%%%%%%%%%%%%%%%%%%%%%%%%%%%%%%%%%%%%%%%%
\begin{frame}{Литература}
    \begin{block}{Набор данных}
    Julia Berezutskaya, Mariska~J. Vansteensel, Erik~J. Aarnoutse, Zachary~V. Freudenburg, Giovanni Piantoni, Mariana~P. Branco, and Nick~F. Ramsey.
    \newblock Open multimodal {iEEG}-{fMRI} dataset from naturalistic stimulation with a short audiovisual film.
    \newblock {\em Scientific Data}, 9(1), March 2022.
    \end{block}

    \begin{block}{Вспомогательные факты}
        Maxim Sharaev, Alexander Andreev, Alexey Artemov, Alexander Bernstein, Evgeny Burnaev, Ekaterina Kondratyeva, Svetlana Sushchinskaya, and Renat Akzhigitov.
        
        \newblock fmri: preprocessing, classification and pattern recognition.
        \newblock {\em Scientific Data}, 2018.
    \end{block}
\end{frame}
%%%%%%%%%%%%%%%%%%%%%%%%%%%%%%%%%%%%%%%%%%%%%%%%%%%%%%%%%%%%%%%%%%%%%%%%%%%%%%%%%%%%%%%%%%%%%%%%%%%%%%%%%%%%%%%%%%%%%%%%%%%%%%%%%%%%%%%%%%%%%%

%%%%%%%%%%%%%%%%%%%%%%%%%%%%%%%%%%%%%%%%%%%%%%%%%%%%%%%%%%%%%%%%%%%%%%%%%%%%%%%%%%%%%%%%%%%%%%%%%%%%%%%%%%%%%%%%%%%%%%%%%%%%%%%%%%%%%%%%%%%%%%
\begin{frame}{Постановка задачи}
Пусть $\bm{\Omega}$~--- видеоряд, $\nu$~--- частота кадров, $t$~--- продолжительность видеоряда:
\begin{equation*}
    \bm{\Omega} = (\bm{\omega}_{1}, \dots, \bm{\omega}_{\nu \cdot t}),
\end{equation*}
где $\bm{\omega} \in \mathbb{R}^{W_{\bm{\omega}} \times H_{\bm{\omega}} \times C_{\bm{\omega}}}$~--- изображение, $W_{\bm{\omega}}$~---
ширина изображения, $H_{\bm{\omega}}$~--- высота изображения и $C_{\bm{\omega}}$~--- число каналов.

$\mathcal{S}$~--- последовательность фМРТ снимков,  $\mu$~--- частота снимков:
\begin{equation*}
    \mathcal{S} = (\bm{s}^{1}, \dots, \bm{s}^{\mu \cdot t}),
\end{equation*}
где $\bm{s} \in \mathbb{R}^{X_{\bm{s}} \times Y_{\bm{s}} \times Z_{\bm{s}}}$~--- фМРТ снимок, $X_{\bm{s}},~Y_{\bm{s}},~Z_{\bm{s}}$~--- размерность одного измерения.

Также считаем, что известно несколько дополнительных измерений фМРТ $\mathcal{S}_0$ того же испытуемого.
Необходимо построить отображение $\bm{f}$:
\begin{equation*}
    \bm{f}(\bm{\omega}_{1}, \dots, \bm{\omega}_{k_i - \nu \cdot \Delta t}, \mathcal{S}_0) = \bm{s}^i,
\end{equation*}
которое учитывает задержку $\Delta t$.
\begin{equation*}
    \label{k_i}
    k_i = \dfrac{\nu \cdot i}{\mu}
\end{equation*}
\end{frame}
%%%%%%%%%%%%%%%%%%%%%%%%%%%%%%%%%%%%%%%%%%%%%%%%%%%%%%%%%%%%%%%%%%%%%%%%%%%%%%%%%%%%%%%%%%%%%%%%%%%%%%%%%%%%%%%%%%%%%%%%%%%%%%%%%%%%%%%%%%%%%%


%%%%%%%%%%%%%%%%%%%%%%%%%%%%%%%%%%%%%%%%%%%%%%%%%%%%%%%%%%%%%%%%%%%%%%%%%%%%%%%%%%%%%%%%%%%%%%%%%%%%%%%%%%%%%%%%%%%%%%%%%%%%%%%%%%%%%%%%%%%%%%
\begin{frame}{Предлагаемая модель}
\begin{equation*}
	\label{main_model}
	\bm{f}(\bm{\omega}_{k_i - \nu \cdot \Delta t}) = \bm{\delta}^i, \ i = 2, \ldots, 641-\mu \cdot \Delta t,
\end{equation*}
\begin{itemize}
    \item $\bm{\delta}^i = \bm{s}^i - \bm{s}^{i-1},~\bm{s}^k \in \mathbb{R}^{40 \times 64 \times 64}$~--- тензор снимка фМРТ под номером $k$.
    \item $\delta_{ijk}$~--- компонента тензора $\bm{\delta}$.
    \item $\bm{x} = [x_1, \ldots, x_{d}]^{T}$~--- вектор признаков изображения, $d=2048$ 
    \item $\bm{w}_{ijk} = [w^{ijk}_1, \ldots, w^{ijk}_{d}]^{T}$~--- вектор весов элемента тензора $\delta_{ijk}$
\end{itemize}
    Для восстановления разности значений в каждом вокселе используется линейная модель
\begin{equation*}
	\label{f_ijk}
	f_{ijk}(\bm{x}, \bm{w}_{ijk}) = \langle \bm{x}, \bm{w}_{ijk} \rangle.
 \end{equation*}
 Для каждого вокселя в снимке задана обучающая выборка:
\begin{equation*}
    \mathcal{D}_{ijk} = \left(\bm{x}_l,~\delta^{l}_{ijk} \right)^{641 - \mu \cdot \Delta t}_{l = 2}
\end{equation*}
$N$~--- число объектов тренировочной выборки. Функция потерь:
\begin{equation*}
	\label{Loss}
	\mathcal{L}_{ijk}(\bm{w}_{ijk}, \Delta t, \alpha) = \frac{1}{2} \sum\limits_{l = 2}^{N+1} \big(f_{ijk}(\bm{x}_l, \bm{w}_{ijk}) - \delta^{l}_{ijk}\big)^2 + \frac{\alpha}{2} \|\bm{w}_{ijk}\|^2_2.
\end{equation*}
\end{frame}
%%%%%%%%%%%%%%%%%%%%%%%%%%%%%%%%%%%%%%%%%%%%%%%%%%%%%%%%%%%%%%%%%%%%%%%%%%%%%%%%%%%%%%%%%%%%%%%%%%%%%%%%%%%%%%%%%%%%%%%%%%%%%%%%%%%%%%%%%%%%%%

%%%%%%%%%%%%%%%%%%%%%%%%%%%%%%%%%%%%%%%%%%%%%%%%%%%%%%%%%%%%%%%%%%%%%%%%%%%%%%%%%%%%%%%%%%%%%%%%%%%%%%%%%%%%%%%%%%%%%%%%%%%%%%%%%%%%%%%%%%%%%%
\begin{frame}{Предлагаемая модель}
Требуется минимизировать функцию потерь $\mathcal{L}_{ijk}(\bm{w}_{ijk}, \Delta t, \alpha)$ при фиксированных гиперпараметрах $\Delta t$ и $\alpha$:
\begin{equation*}
	\label{main_Problem}
	\hat{\bm{w}}_{ijk} = \argmin_{\bm{w}_{ijk}} \mathcal{L}_{ijk}(\bm{w}_{ijk}, \Delta t, \alpha).
\end{equation*}

Определим матрицу объектов тренировочной выборки
\begin{equation*}
\label{X}
    \bm{X} = [\bm{x}_2^T, \dots, \bm{x}_{N+1}^T]^T = [x^i_j] \in \mathbb{R}^{N \times d}
\end{equation*}
и вектор, компонентами которого являются разности значений одного и того же вокселя в снимках тренировочной выборки,
\begin{equation*}
\label{Delta}
    \bm{\Delta}_{ijk} = [\delta^{2}_{ijk}, \dots, \delta^{N+1}_{ijk}]^T \in \mathbb{R}^{N}.
\end{equation*}
Тогда решение можно записать в виде:
\begin{equation*}
\label{weights}
    \hat{\bm{w}}_{ijk} = (\bm{X}^T \bm{X} + \alpha \mathbf{I})^{-1} \bm{X}^T \bm{\Delta}_{ijk}.
\end{equation*}

\end{frame}
%%%%%%%%%%%%%%%%%%%%%%%%%%%%%%%%%%%%%%%%%%%%%%%%%%%%%%%%%%%%%%%%%%%%%%%%%%%%%%%%%%%%%%%%%%%%%%%%%%%%%%%%%%%%%%%%%%%%%%%%%%%%%%%%%%%%%%%%%%%%%%
\begin{frame}{Вычислительный эксперимент}
\begin{block}{Цели эксперимента}
\begin{enumerate}
    \item Проверка работоспособности и корректности предложенного метода
    \item Проверка предположения о наличии зависимости между показаниями датчиков фМРТ и восприятием внешнего мира человеком
    \item Проверка предположения о наличии задержки $\Delta t$
\end{enumerate}
\end{block}
\begin{block}{Выборка} 
Набор данных включает в себя в себя записи фМРТ 30 участников в возрасте от 7 до 47 лет во время выполнения одинаковой задачи и записи внутричерепной электроэнцефалографии 51 участникa в возрасте от 5 до 55 лет. 
\end{block}

\end{frame}
%%%%%%%%%%%%%%%%%%%%%%%%%%%%%%%%%%%%%%%%%%%%%%%%%%%%%%%%%%%%%%%%%%%%%%%%%%%%%%%%%%%%%%%%%%%%%%%%%%%%%%%%%%%%%%%%%%%%%%%%%%%%%%%%%%%%%%%%%%%%



\begin{frame}{Анализ ошибки}
\begin{block}{Варьирование гиперпараметра $\Delta t$}
\begin{figure}[h!]
    \centering
    \includegraphics[width=0.46\textwidth]{subs_delta_MSE_dt.pdf}
    \label{MSE_dt_main}
\end{figure}

График демонстрирует наличие зависимости результата предсказания от гиперпараметра $\Delta t$. 
Причем при нефизичной задержке больше десяти секунд ошибка начинает расти. Минимум достигается в пределах 4~---10 секунд.
\end{block}
\end{frame}
%%%%%%%%%%%%%%%%%%%%%%%%%%%%%%%%%%%%%%%%%%%%%%%%%%%%%%%%%%%%%%%%%%%%%%%%%%%%%%%%%%%%%%%%%%%%%%%%%%%%%%%%%%%%%%%%%%%%%%%%%%%%%%%%%%%%%%%%%%%%%%%%%%%%%%%%%%%%%%%%%%%%%%%%%%%%%%%%%%%%%%%%%%%%%%%%%%%%%%%%%%%%%%%%%%

\begin{frame}{Анализ ошибки}
\begin{block}{Варьирование гиперпараметра $\alpha$}
Проанализированна зависимость MSE от коэффициента регуляризации $\alpha$.
Рассматриваются коэффициенты сжатия 1, 2, 4 и 8. 
\begin{figure}[h!]
    \centering
    \includegraphics[width=0.46\textwidth]{subs_MSE_alpha.pdf}
    \label{subs_MSE_alpha}
\end{figure}
Из графиков видно, что отстуствие регуляризации ведет к переобучению модели.
Оптимальное значение коэффициента $\alpha \approx 100$.
\end{block}
\end{frame}
%%%%%%%%%%%%%%%%%%%%%%%%%%%%%%%%%%%%%%%%%%%%%%%%%%%%%%%%%%%%%%%%%%%%%%%%%%%%%%%%%%%%%%%%%%%%%%%%%%%%%%%%%%%%%%%%%%%%%%%%%%%%%%%%%%%%%%%%%%%%%%%%%%%%%%%%%%%%%%%%%%%%%%%%%%%%%%%%%%%%%%%%%%%%%%%%%%%%%%%%%%%%%%%%%%

\begin{frame}{Анализ ошибки}

В качестве демонстрации работы алгоритма на рисунке ниже приведены срезы оригинального и предсказанного воксельного снимка фМРТ из тестовой выборки. 
\begin{figure}[h!]
    \centering
    \subfloat[Истинный]{\label{fig:5a}{\includegraphics[width=0.33\textwidth]{sub-04-5-1-1000-100-20-_-_-test.pdf}}}
    \hfill
    \subfloat[Восстановленный]{\label{fig:5b}{\includegraphics[width=0.33\textwidth]{sub-04-5-1-1000-100-20-_-_-predicted.pdf}}}
    \hfill
    \subfloat[Разность]{\label{fig:5c}{\includegraphics[width=0.33\textwidth]{sub-04-5-1-1000-100-20-_-_-difference.pdf}}}
    \label{fig:5}
\end{figure}
Значения в вокселях нормализованны на отрезок $[0,~1]$, поэтому ошибка порядка $10^{-3}$ говорит о достаточно хорошем качестве работы алгоритма.

\end{frame}
%%%%%%%%%%%%%%%%%%%%%%%%%%%%%%%%%%%%%%%%%%%%%%%%%%%%%%%%%%%%%%%%%%%%%%%%%%%%%%%%%%%%%%%%%%%%%%%%%%%%%%%%%%%%%%%%%%%%%%%%%%%%%%%%%%%%%%%%%%%%%%%%%%%%%%%%%%%%%%%%%%%%%%%%%%%%%%%%%%%%%%%%%%%%%%%%%%%%%%%%%%%%%%%%%%%%
\begin{frame}{Анализ ошибки}
\begin{block}{Распределения значений компонент вектора весов модели.}
Для построения производилось усреднение по строкам матрицы весов $\hat{W}$, то есть усреднение по всем вокселям.
\begin{figure}[h!]
    \centering
    \includegraphics[width=0.46\textwidth]{mean_weight_distribution.pdf}
    \label{fig:6}
\end{figure}
График хорошо аппроксимируется плотнотью нормального распределения, что говорит о статистической значимости весов.
\end{block}

\end{frame}
%%%%%%%%%%%%%%%%%%%%%%%%%%%%%%%%%%%%%%%%%%%%%%%%%%%%%%%%%%%%%%%%%%%%%%%%%%%%%%%%%%%%%%%%%%%%%%%%%%%%%%%%%%%%%%%%%%%%%%%%%%%%%%%%%%%%%%%%%%%%%%%%%%%%%%%%%%%%%%%%%%%%%%%%%%%%%%%%%%%%%%%%%%%%%%%%%%%%%%%%%%%%%%%%%%%%
\begin{frame}{Анализ ошибки: инвариантность весов модели относительно человека}

Экспериментально проверено, что модель улавливает общие для всех испытуемых зависимости между данными.
Другими словами, восстановление снимка фМРТ одного человека можно производить, используя матрицу весов другого испытуемого.
Результаты представлены в таблице ниже:

	\begin{table}[h!]
		\centering
		\begin{tabular}{|c|c|c|}
			\hline
			Матрица весов	&	Истинная	&	Подмешанная \\ \hline \hline
			MSE		& 	$1.02 \cdot 10^{-4}$	 &		$1.05 \cdot 10^{-4}$ \\ \hline
		\end{tabular}
		\label{table_2}
	\end{table}
MSE почти совпадают, что говорит о справедливости рассмотренного предположения.

\end{frame}
%%%%%%%%%%%%%%%%%%%%%%%%%%%%%%%%%%%%%%%%%%%%%%%%%%%%%%%%%%%%%%%%%%%%%%%%%%%%%%%%%%%%%%%%%%%%%%%%%%%%%%%%%%%%%%%%%%%%%%%%%%%%%%%%%%%%%%%%%%%%%%%%%%%%%%%%%%%%%%%%%%%%%%%%%%%%%%%%%%%%%%%%%%%%%%%%%%%%%%%%%%%%%%%%%%%%
\begin{frame}{Анализ ошибки: результаты работы модели на случайном шуме}

Первоначально модель обучена на оригинальных изображениях из видеоряда.
Получена соответствующая матрица весов $\hat{W}$. 
К первому снимку последовательно прибавляются все восстановленные изменения значений вокселей. 
В результате имеем последний снимок последовательности:
\begin{figure}[h!]
    \centering
    \subfloat[Истинный]{\label{fig:7a}{\includegraphics[width=0.33\textwidth]{sub-35-5-1-1000--1-20-_-_-recovered-test.pdf}}}
    \hfill
    \subfloat[Восстановленный]{\label{fig:7b}{\includegraphics[width=0.33\textwidth]{sub-35-5-1-1000--1-20-_-_-recovered-predicted.pdf}}}
    \hfill
    \subfloat[Разность]{\label{fig:7c}{\includegraphics[width=0.33\textwidth]{sub-35-5-1-1000--1-20-_-_-recovered-difference.pdf}}}
    \caption{Срез снимка фМРТ из тестовой выборки}
    \label{fig:7}
\end{figure}
\end{frame}
%%%%%%%%%%%%%%%%%%%%%%%%%%%%%%%%%%%%%%%%%%%%%%%%%%%%%%%%%%%%%%%%%%%%%%%%%%%%%%%%%%%%%%%%%%%%%%%%%%%%%%%%%%%%%%%%%%%%%%%%%%%%%%%%%%%%%%%%%%%%%%%%%%%%%%%%%%%%%%%%%%%%%%%%%%%%%%%%%%%%%%%%%%%%%%%%%%%%%%%%%%%%%%%%%%%%
\begin{frame}{Анализ ошибки: результаты работы модели на случайном шуме}

Для демонстрации работы на случайном шуме сгенерирована случайная выборка из векторов признакового описания изображения размера тестовой выборки.
По шумовым данным и матрице весов $\hat{W}$ получена последовательность изменений между соседними снимками фМРТ. 
В результате аналогично предсказанию в случае оригинальных изображений имеем последний снимок последовательности, получаенный по шумовым данным:
\begin{figure}[h!]
    \centering
    \subfloat[Истинный]{\label{fig:8a}{\includegraphics[width=0.33\textwidth]{sub-35-5-1-1000--1-20-_-_-recovered-test_noised.pdf}}}
    \hfill
    \subfloat[Восстановленный по шуму]{\label{fig:8b}{\includegraphics[width=0.33\textwidth]{sub-35-5-1-1000--1-20-_-_-recovered-predicted_noised.pdf}}}
    \hfill
    \subfloat[Разность]{\label{fig:8c}{\includegraphics[width=0.33\textwidth]{sub-35-5-1-1000--1-20-_-_-recovered-difference_noised.pdf}}}
    \label{fig:8}
\end{figure}
\end{frame}
%%%%%%%%%%%%%%%%%%%%%%%%%%%%%%%%%%%%%%%%%%%%%%%%%%%%%%%%%%%%%%%%%%%%%%%%%%%%%%%%%%%%%%%%%%%%%%%%%%%%%%%%%%%%%%%%%%%%%%%%%%%%%%%%%%%%%%%%%%%%%%%%%%%%%%%%%%%%%%

\begin{frame}{Анализ ошибки: результаты работы модели на случайном шуме}
\begin{block}{}
В таблице приведены среднеквадратичные ошибки в случае истинной выборки и в случае выборки со случайным шумом.

\begin{table}[h!]
    \centering
    \begin{tabular}{|c|c|c|}
        \hline
        Выборка	&	Истинная	&	Случайный шум \\ \hline \hline
        MSE		& 	$2 \cdot 10^{-3}$	 &		$10^{-1}$ \\ \hline
    \end{tabular}
    \label{table_3}
\end{table}

Ошибка на шуме на порядок больше, что подтверждает наличие зависимости между показаниями датчиков и изображениями из видеоряда.
\end{block}
\end{frame}
%%%%%%%%%%%%%%%%%%%%%%%%%%%%%%%%%%%%%%%%%%%%%%%%%%%%%%%%%%%%%%%%%%%%%%%%%%%%%%%%%%%%%%%%%%%%%%%%%%%%%%%%%%%%%%%%%%%%%%%%%%%%%%%%%%%%%%%%%%%%%%%%%%%%%%%%%%%%%%%%%%%%%%%%%%%%%%%%%%%%%%%%%%%%%%%%%%%%%%%%%%%%%%%%%%%%
\begin{frame}{Выводы}
\begin{itemize}
    \item В работе построен метод аппроксимации показаний датичков фМРТ по видеоряду, просматриваемому человеком. 
    \item Проверено, что качество работы модели на случайном шуме горазда хуже, чем на оригинальных изображениях из видеоряда.
    \item Проверено, что веса модели инвариантны относительно человека
    \item Результаты экспериментов подтверждают наличие зависимости между показаниями датчиков фМРТ и восприятием внешнего мира человеком.
    \item Приведенные графики подтверждают предположение о наличии задержки между моментом получения информации зрительными органами и реакцией мозга на эту информацию.
\end{itemize}
\end{frame}
\end{document}