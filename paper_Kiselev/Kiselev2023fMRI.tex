\documentclass[a4paper, 12pt]{article}

\usepackage{arxiv}

\usepackage[T2A]{fontenc}
\usepackage[utf8]{inputenc}
\usepackage[english, russian]{babel}
% \usepackage{cmap}
\usepackage{url}
\usepackage{booktabs}
\usepackage{nicefrac}
\usepackage{microtype}
\usepackage{lipsum}
\usepackage{graphicx}
\usepackage{epstopdf}
\usepackage{subfig}
\usepackage[square,sort,comma,numbers]{natbib}
\usepackage{doi}
\usepackage{multicol}
\usepackage{multirow}
\usepackage{tabularx}
\usepackage{float}

\usepackage{tikz}
\usetikzlibrary{matrix}

% Algorithms
\usepackage{algpseudocode}
\usepackage{algorithm}

%% Шрифты
\usepackage{euscript} % Шрифт Евклид
\usepackage{mathrsfs} % Красивый матшрифт
\usepackage{extsizes} % Возможность сделать 14-й шрифт
\usepackage{bm}

\usepackage{makecell} % diaghead in a table
\usepackage{amsmath,amsfonts,amssymb,amsthm,mathtools,dsfont}
\usepackage{icomma}
\usepackage[labelfont=bf]{caption}
\usepackage{subfig} % for subfigures
\usepackage{wrapfig}

\newcommand{\bz}{\mathbf{z}}
\newcommand{\bx}{\mathbf{x}}
\newcommand{\by}{\mathbf{y}}
\newcommand{\bv}{\mathbf{v}}
\newcommand{\bw}{\mathbf{w}}
\newcommand{\ba}{\mathbf{a}}
\newcommand{\bb}{\mathbf{b}}
\newcommand{\bp}{\mathbf{p}}
\newcommand{\bq}{\mathbf{q}}
\newcommand{\bt}{\mathbf{t}}
\newcommand{\bu}{\mathbf{u}}
\newcommand{\bs}{\mathbf{s}}
\newcommand{\bT}{\mathbf{T}}
\newcommand{\bX}{\mathbf{X}}
\newcommand{\bZ}{\mathbf{Z}}
\newcommand{\bS}{\mathbf{S}}
\newcommand{\bH}{\mathbf{H}}
\newcommand{\bW}{\mathbf{W}}
\newcommand{\bY}{\mathbf{Y}}
\newcommand{\bU}{\mathbf{U}}
\newcommand{\bQ}{\mathbf{Q}}
\newcommand{\bP}{\mathbf{P}}
\newcommand{\bA}{\mathbf{A}}
\newcommand{\bB}{\mathbf{B}}
\newcommand{\bC}{\mathbf{C}}
\newcommand{\bE}{\mathbf{E}}
\newcommand{\bF}{\mathbf{F}}
\newcommand{\bomega}{\boldsymbol{\omega}}
\newcommand{\btheta}{\boldsymbol{\theta}}
\newcommand{\bgamma}{\boldsymbol{\gamma}}
\newcommand{\bdelta}{\boldsymbol{\delta}}
\newcommand{\bPsi}{\boldsymbol{\Psi}}
\newcommand{\bpsi}{\boldsymbol{\psi}}
\newcommand{\bxi}{\boldsymbol{\xi}}
\newcommand{\bchi}{\boldsymbol{\chi}}
\newcommand{\bzeta}{\boldsymbol{\zeta}}
\newcommand{\blambda}{\boldsymbol{\lambda}}
\newcommand{\beps}{\boldsymbol{\varepsilon}}
\newcommand{\bZeta}{\boldsymbol{Z}}
% mathcal
\newcommand{\cX}{\mathcal{X}}
\newcommand{\cY}{\mathcal{Y}}
\newcommand{\cW}{\mathcal{W}}

\newcommand{\dH}{\mathds{H}}
\newcommand{\dR}{\mathds{R}}
% transpose
\newcommand{\T}{^{\mathsf{T}}}

% \renewcommand{\shorttitle}{\textit{arXiv} Шаблон}
\renewcommand{\epsilon}{\ensuremath{\varepsilon}}
\renewcommand{\phi}{\ensuremath{\varphi}}
\renewcommand{\kappa}{\ensuremath{\varkappa}}
\renewcommand{\le}{\ensuremath{\leqslant}}
\renewcommand{\leq}{\ensuremath{\leqslant}}
\renewcommand{\ge}{\ensuremath{\geqslant}}
\renewcommand{\geq}{\ensuremath{\geqslant}}
\renewcommand{\emptyset}{\varnothing}

\DeclareMathOperator*{\argmax}{arg\,max}  % in your preamble
\DeclareMathOperator*{\argmin}{arg\,min}  % in your preamble 

\usepackage{hyperref}
% \usepackage[usenames,dvipsnames,svgnames,table,rgb]{xcolor}

\hypersetup{
	unicode=true,
	colorlinks=true,
	linkcolor=black,        % внутренние ссылки
	citecolor=blue,         % на библиографию
	filecolor=magenta,      % на файлы
	urlcolor=blue           % на URL
}

\graphicspath{{./figures}}

\usepackage{enumitem} % Для модификаций перечневых окружений

\theoremstyle{definition} % "Определение"
\newtheorem{definition}{Опр.}[section]

\usepackage{etoolbox}

\makeatletter
\expandafter\patchcmd\csname\string\algorithmic\endcsname{\itemsep\z@}{\itemsep=1.5mm}{}{}
\makeatother

\newcommand{\myfigref}[2]{~\ref{#1}.\subref{#2}}% <---- a new macro for referring to a subfigure
\renewcommand{\abstractname}{Аннотация}

\title{Моделирование показания fMRI по видео, показанному человеку}

\author{
	Дорин Даниил \\
	\texttt{dorin.dd@phystech.edu} \\
	\And
	Киселев Никита \\
	\texttt{kiselev.ns@phystech.edu} \\
}
\date{\today}

\begin{document}
\maketitle

\begin{abstract}
	
	В работе исследуется проблема восстановления зависимости между показаниями датчиков fMRI
	и восприятием внешнего мира человеком.
	Проверяется гипотеза зависимости между последовательностью fMRI-снимков и видеорядом,
	просматриваемым человеком. 
	Предлагается метод аппроксимации показаний fMRI по просматриваемому видеоряду.
	В качестве демонстрации результатов работы проводятся эксперименты на 
	выборке~\citep{Berezutskaya2022}, полученной при обследовании большого числа испытуемых.

\end{abstract}


\keywords{нейровизуализация \and fMRI \and видеоряд \and зависимость между данными}

\section{Введение}

	Сококупность методов, позволяющих визуализировать структуру и функции человеческого мозга,
	называется \textit{нейровизуализацией}. Эти методы использутся для изучения мозга, а также
	для обнаружения заболеваний и психических расстройств. Одним из самых активно развивающихся
	видов нейровизуализации является fMRI.
	
	\textit{Функциональная магнитно-резонансная томография} или \textit{фМРТ} (англ.~\textit{fMRI}) 
	является разновидностью магнитно-резонансной томографии и основана на изменениях в токе крови, 
	вызванных нейронной активностью мозга. Изображения, получаемые с помощью fMRI, показывают,
	какие участки мозга активированы при выполнении испытуемым определенных заданий.

	Несмотря на большую ценность fMRI для нейровизуализации, этот метод имеет ряд важных ограничений.
	Так, в работах \citep{menon1999spatial, logothetis2008we} подробно рассматриваются 
	пространственное и временное разрешения fMRI, последнее из которых является его существенным
	недостатком. Кроме того, при томографическом обследовании неизбежно возникают шумы, 
	связанные с движением объекта в сканере, сердцебиением и дыханием человека, тепловыми
	флуктуациями самого прибора и т.\,д. В работе \citep{1804.10167} предлагаются методы 
	подавления вышеперечисленных шумов на основе графов и демонстрируется их эффективность в задаче
	выявления эпилепсии и депрессии.

	%В настоящей работе рассматривается задача восстановления зависимости между последовательностью
	%fMRI-снимков и видеорядом, который в этот момент просматривает испытуемый. 

	%В работе исследуется зависимость между данными, представленными в работе \citep{Berezutskaya2022}. 
	%Этот набор данных был получен при томографическом обследовании большой группы испытуемых (30 участников, возрастной диапазон~--- от 7 до 47 лет),
	%которым предлагалось выполнить одно и то же задание~--- просмотреть короткий аудиовизуальный
	%фильм. 
	
	% существующих работах по рассматриваемой теме предлагаются различные модели аппроксимации 
	%показаний fMRI, в частности, архитектура глубоких нейронных сетей трансформер. Тем не менее,
	%на текущий момент проблема построения и тестирования такой модели остается открытой.
	%С этой целью в данной работе предлагается проверить гипотезу зависимости между данными, 
	%используя статистические критерии.

	Данные, на которых проводится проверка гипотезы зависимости и анализ работы построенного метода,
	представлены в работе \citep{Berezutskaya2022}. Этот набор данных был получен при
	fMRI-обследовании большой группы людей. Им предлагалось выполнить одно и то же задание~---
	просмотреть короткий аудиовизуальный фильм. Для него в рассматриваемой работе были 
	сгенерированы аннотации, содержащие в том числе информацию о времени появления и исчезновения
	отдельных слов, объектов и персонажей. Методы аудио- и видеоаннотирования подробно излагаются в
	\citep{boersma2018praat} и \citep{Berezutskaya2020}. 

\section{Постановка задачи}

\section{Вычислительный эксперимент}

\section{Анализ ошибки}

\section{Заключение}

\bibliographystyle{plain}
\bibliography{Kiselev2023fMRI.bib}

\end{document}