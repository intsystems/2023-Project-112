\documentclass[a4paper, 12pt]{article}

\usepackage{arxiv}

\usepackage[T2A]{fontenc}
\usepackage[utf8]{inputenc}
\usepackage[english, russian]{babel}
% \usepackage{cmap}
\usepackage{url}
\usepackage{booktabs}
\usepackage{nicefrac}
\usepackage{microtype}
\usepackage{lipsum}
\usepackage{graphicx}
\usepackage{epstopdf}
\usepackage{subfig}
\usepackage[square,sort,comma,numbers]{natbib}
\usepackage{doi}
\usepackage{multicol}
\usepackage{multirow}
\usepackage{tabularx}
\usepackage{float}

\usepackage{tikz}
\usetikzlibrary{matrix}

% Algorithms
\usepackage{algpseudocode}
\usepackage{algorithm}

%% Шрифты
\usepackage{euscript} % Шрифт Евклид
\usepackage{mathrsfs} % Красивый матшрифт
\usepackage{extsizes} % Возможность сделать 14-й шрифт
\usepackage{bm}

\usepackage{makecell} % diaghead in a table
\usepackage{amsmath,amsfonts,amssymb,amsthm,mathtools,dsfont}
\usepackage{icomma}
\usepackage[labelfont=bf]{caption}
\usepackage{subfig} % for subfigures
\usepackage{wrapfig}

\newcommand{\bz}{\mathbf{z}}
\newcommand{\bx}{\mathbf{x}}
\newcommand{\by}{\mathbf{y}}
\newcommand{\bv}{\mathbf{v}}
\newcommand{\bw}{\mathbf{w}}
\newcommand{\ba}{\mathbf{a}}
\newcommand{\bb}{\mathbf{b}}
\newcommand{\bp}{\mathbf{p}}
\newcommand{\bq}{\mathbf{q}}
\newcommand{\bt}{\mathbf{t}}
\newcommand{\bu}{\mathbf{u}}
\newcommand{\bs}{\mathbf{s}}
\newcommand{\bT}{\mathbf{T}}
\newcommand{\bX}{\mathbf{X}}
\newcommand{\bZ}{\mathbf{Z}}
\newcommand{\bS}{\mathbf{S}}
\newcommand{\bH}{\mathbf{H}}
\newcommand{\bW}{\mathbf{W}}
\newcommand{\bY}{\mathbf{Y}}
\newcommand{\bU}{\mathbf{U}}
\newcommand{\bQ}{\mathbf{Q}}
\newcommand{\bP}{\mathbf{P}}
\newcommand{\bA}{\mathbf{A}}
\newcommand{\bB}{\mathbf{B}}
\newcommand{\bC}{\mathbf{C}}
\newcommand{\bE}{\mathbf{E}}
\newcommand{\bF}{\mathbf{F}}
\newcommand{\bomega}{\boldsymbol{\omega}}
\newcommand{\btheta}{\boldsymbol{\theta}}
\newcommand{\bgamma}{\boldsymbol{\gamma}}
\newcommand{\bdelta}{\boldsymbol{\delta}}
\newcommand{\bPsi}{\boldsymbol{\Psi}}
\newcommand{\bpsi}{\boldsymbol{\psi}}
\newcommand{\bxi}{\boldsymbol{\xi}}
\newcommand{\bchi}{\boldsymbol{\chi}}
\newcommand{\bzeta}{\boldsymbol{\zeta}}
\newcommand{\blambda}{\boldsymbol{\lambda}}
\newcommand{\beps}{\boldsymbol{\varepsilon}}
\newcommand{\bZeta}{\boldsymbol{Z}}
% mathcal
\newcommand{\cX}{\mathcal{X}}
\newcommand{\cY}{\mathcal{Y}}
\newcommand{\cW}{\mathcal{W}}

\newcommand{\dH}{\mathds{H}}
\newcommand{\dR}{\mathds{R}}
% transpose
\newcommand{\T}{^{\mathsf{T}}}

% \renewcommand{\shorttitle}{\textit{arXiv} Шаблон}
\renewcommand{\epsilon}{\ensuremath{\varepsilon}}
\renewcommand{\phi}{\ensuremath{\varphi}}
\renewcommand{\kappa}{\ensuremath{\varkappa}}
\renewcommand{\le}{\ensuremath{\leqslant}}
\renewcommand{\leq}{\ensuremath{\leqslant}}
\renewcommand{\ge}{\ensuremath{\geqslant}}
\renewcommand{\geq}{\ensuremath{\geqslant}}
\renewcommand{\emptyset}{\varnothing}

\DeclareMathOperator*{\argmax}{arg\,max}  % in your preamble
\DeclareMathOperator*{\argmin}{arg\,min}  % in your preamble 

\usepackage{hyperref}
% \usepackage[usenames,dvipsnames,svgnames,table,rgb]{xcolor}

\hypersetup{
	unicode=true,
	colorlinks=true,
	linkcolor=black,        % внутренние ссылки
	citecolor=blue,         % на библиографию
	filecolor=magenta,      % на файлы
	urlcolor=blue           % на URL
}

\graphicspath{{./figures}}

\usepackage{enumitem} % Для модификаций перечневых окружений

\theoremstyle{definition} % "Определение"
\newtheorem{definition}{Опр.}[section]

\usepackage{etoolbox}

\makeatletter
\expandafter\patchcmd\csname\string\algorithmic\endcsname{\itemsep\z@}{\itemsep=1.5mm}{}{}
\makeatother

\newcommand{\myfigref}[2]{~\ref{#1}.\subref{#2}}% <---- a new macro for referring to a subfigure
\renewcommand{\abstractname}{Аннотация}

\title{Моделирование показания fMRI по видео, показанному человеку}

\author{
	%Дорин Даниил \\
	%\texttt{dorin.dd@phystech.edu} \\
	%\And
	Киселев Никита \\
	\texttt{kiselev.ns@phystech.edu} \\
	\And
	Грабовой Андрей \\
	\texttt{grabovoy.av@phystech.edu}
}
\date{\today}

\begin{document}
\maketitle

\begin{abstract}
	
	В работе исследуется проблема восстановления зависимости между показаниями датчиков fMRI
	и восприятием внешнего мира человеком.
	%Проверяется гипотеза зависимости между последовательностью fMRI-снимков и видеорядом,
	%просматриваемым человеком. 
	%Предлагается метод аппроксимации показаний fMRI по просматриваемому видеоряду.
	Проводится анализ зависимости между последовательностью fMRI-снимков и видеорядом,
	просматриваемым человеком.
	На основе исследования зависимости предлагается метод аппроксимации показаний fMRI по
	просматриваемому видеоряду.
	Для анализа предложенного метода проводится вычислительный эксперимент на 
	выборке, полученной при томографическом обследовании большого числа испытуемых.

\end{abstract}


\keywords{нейровизуализация \and fMRI \and видеоряд \and зависимость между данными \and Transformer}

\section{Введение}

	Сококупность методов, позволяющих визуализировать структуру и функции человеческого мозга,
	называется \textit{нейровизуализацией}. Эти методы использутся для изучения мозга, а также
	для обнаружения заболеваний и психических расстройств. 
	%Одним из самых активно развивающихся видов нейровизуализации является fMRI.
	Примерами таких являются fMRI и iEEG \citep{Berezutskaya2022}.
	В настоящей работе рассматривается метод~fMRI.

	\textit{Функциональная магнитно-резонансная томография} или \textit{фМРТ} (англ.~\textit{fMRI}) 
	является разновидностью магнитно-резонансной томографии и основана на изменениях в токе крови, 
	вызванных нейронной активностью мозга \citep{Glover2011}. 
	Эти изменения происходят не моментально, а с некоторой задержкой.
	Она возникает из-за того, что сосудистая система достаточно долго реагирует 
	на потребность мозга в глюкозе \citep{Logothetis2003}. Изображения, получаемые с помощью fMRI,
	показывают, какие участки мозга активированы при выполнении испытуемым определенных заданий.

	Метод fMRI играет большую роль в нейровизуализации, однако имеет ряд важных ограничений.
	В работах \citep{menon1999spatial, logothetis2008we} рассматриваются 
	временное и пространственное разрешения fMRI. Временное разрешение является существенным
	недостатком данного метода. Другой недостаток fMRI~--- неизбежно возникающие шумы, 
	связанные с движением объекта в сканере, сердцебиением и дыханием человека, тепловыми
	флуктуациями самого прибора и т.\,д. В работе \citep{1804.10167} предлагаются методы 
	подавления вышеперечисленных шумов на основе графов и демонстрируется их эффективность в задаче
	выявления эпилепсии и депрессии.

	Обобщением уже естественных для обработки изображений 2D сверток в CNN являются 3D
	свертки \citep{Tran_2015_ICCV}.
	Они агрегируют информацию как по времени, так и по пространству.
	Однако это приводит к сильному увеличению количества используемых параметров.
	В~настоящей работе используется наиболее современная архитектура~--- Transformer.
	Впервые она была предложена в статье \citep{https://doi.org/10.48550/arxiv.1706.03762}.
	Не так давно появилась адаптация архитектуры Transformer для работы с видео
	\citep{https://doi.org/10.48550/arxiv.2201.04288}. Данная архитектура состоит из кодировщика
	и декодировщика, каждый из которых в свою очередь состоит из отдельных слоев. Использование 
	механизма Attention \citep{https://doi.org/10.48550/arxiv.1706.03762} 
	позволяет значительно повысить качество работы модели.

	Настоящая работа посвящена восстановлению зависимости между fMRI-снимками и видеорядом.
	Используется предположение, что такая зависимость существует.
	Кроме того, предполагается, что между снимком и видеорядом есть постоянная задержка во времени
	\citep{Logothetis2003}.
	На основе анализа зависимости предлагается метод аппроксимации показаний fMRI по
	просматриваемому видеоряду.

	Данные, на которых проводятся проверка гипотезы зависимости и демонстрация работы построенного 
	метода, представлены в работе \citep{Berezutskaya2022}. Этот набор данных был получен при
	fMRI-обследовании большой группы людей. Им предлагалось выполнить одно и то же задание~---
	просмотреть короткий аудиовизуальный фильм. Для него в рассматриваемой работе были 
	сгенерированы аннотации, содержащие в том числе информацию о времени появления и исчезновения
	отдельных слов, объектов и персонажей. Методы аудио- и видеоаннотирования подробно излагаются в
	\citep{boersma2018praat} и \citep{Berezutskaya2020}. 

\section{Постановка задачи}

	Пусть задана частота кадров $\nu$ и продолжительность $t$ видеоряда. 
	Задан видеоряд
	\begin{equation}
		\mathbf{V} = \{\bs_1, \ldots, \bs_{\nu t}\},
	\end{equation}
	где $\bs_i \in \mathbb{R}^{W_{\mathbf{V}} \times H_{\mathbf{V}} \times C_{\mathbf{V}}},
	\ i = 1, \ldots, \nu t$~--- изображение, а $W_{\mathbf{V}}, H_{\mathbf{V}}$ и 
	$C_{\mathbf{V}}$~--- его ширина, высота и число каналов соответственно.
	Задана последовательность fMRI-снимков
	\begin{equation}
		\mathbf{F} = \{\mathbf{f}_1, \ldots, \mathbf{f}_n\},
	\end{equation}
	где $\mathbf{f}_i \in \mathbb{R}^{W_{\mathbf{F}} \times H_{\mathbf{F}} \times D_{\mathbf{F}}},
	\ i = 1, \ldots, n$~--- томографическое изображение с измерениями
	$W_{\mathbf{F}}, H_{\mathbf{F}}$ и $D_{\mathbf{F}}$. 

	Задача состоит в построении отображения, которое бы учитывало задержку $\Delta t$ между
	fMRI-снимком и видеорядом, а также предыдущие томографические показания. Формально, необходимо
	найти такое отображение $g$, что
	\begin{equation}
		g(\bs_1, \ldots, \bs_{k_i - \nu \Delta t}; \mathbf{f}_1, \ldots, \mathbf{f}_{i-1}) = \mathbf{f}_i,
		\ i = 1, \ldots, n.
	\end{equation}

\section{Вычислительный эксперимент}

\section{Анализ ошибки}

\section{Заключение}

\bibliographystyle{plain}
\bibliography{Kiselev2023fMRI.bib}

\end{document}