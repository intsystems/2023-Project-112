\documentclass[a4paper,12pt]{article}

%%% Работа с русским языком
\usepackage{cmap}					% поиск в PDF
\usepackage{mathtext} 				% русские буквы в формулах
\usepackage[T2A]{fontenc}			% кодировка
\usepackage[utf8]{inputenc}			% кодировка исходного текста
\usepackage[english,russian]{babel}	% локализация и переносы
\usepackage{indentfirst}
\frenchspacing


%%% Дополнительная работа с математикой
\usepackage{amsmath,amsfonts,amssymb,amsthm,mathtools} % AMS
\usepackage{icomma} % "Умная" запятая: $0,2$ --- число, $0, 2$ --- перечисление

%% Номера формул
%\mathtoolsset{showonlyrefs=true} % Показывать номера только у тех формул, на которые есть \eqref{} в тексте.
%\usepackage{leqno} % Нумерация формул слева

%% Свои команды
\DeclareMathOperator{\sgn}{\mathop{sgn}}

%% Перенос знаков в формулах (по Львовскому)
\newcommand*{\hm}[1]{#1\nobreak\discretionary{}
	{\hbox{$\mathsurround=0pt #1$}}{}}

%%% Работа с картинками
\usepackage{graphicx}  % Для вставки рисунков
\graphicspath{{images/}}  % папки с картинками
\setlength\fboxsep{3pt} % Отступ рамки \fbox{} от рисунка
\setlength\fboxrule{1pt} % Толщина линий рамки \fbox{}
\usepackage{wrapfig} % Обтекание рисунков текстом

%%% Работа с таблицами
\usepackage{array,tabularx,tabulary,booktabs} % Дополнительная работа с таблицами
\usepackage{longtable}  % Длинные таблицы
\usepackage{multirow} % Слияние строк в таблице

%%% Теоремы
\theoremstyle{plain} % Это стиль по умолчанию, его можно не переопределять.
\newtheorem{theorem}{Теорема}[section]
\newtheorem{proposition}[theorem]{Утверждение}

\theoremstyle{definition} % "Определение"
\newtheorem{corollary}{Следствие}[theorem]
\newtheorem{problem}{Задача}[section]

\theoremstyle{remark} % "Примечание"
\newtheorem*{nonum}{Решение}

%%% Программирование
\usepackage{etoolbox} % логические операторы

%%% Страница
\usepackage{extsizes} % Возможность сделать 14-й шрифт
\usepackage{geometry} % Простой способ задавать поля
\geometry{top=15mm}
\geometry{bottom=20mm}
\geometry{left=20mm}
\geometry{right=20mm}
%
%\usepackage{fancyhdr} % Колонтитулы
% 	\pagestyle{fancy}
%\renewcommand{\headrulewidth}{0pt}  % Толщина линейки, отчеркивающей верхний колонтитул
% 	\lfoot{Нижний левый}
% 	\rfoot{Нижний правый}
% 	\rhead{Верхний правый}
% 	\chead{Верхний в центре}
% 	\lhead{Верхний левый}
%	\cfoot{Нижний в центре} % По умолчанию здесь номер страницы

\usepackage{setspace} % Интерлиньяж
%\onehalfspacing % Интерлиньяж 1.5
%\doublespacing % Интерлиньяж 2
%\singlespacing % Интерлиньяж 1

\usepackage{lastpage} % Узнать, сколько всего страниц в документе.

\usepackage{soul} % Модификаторы начертания

\usepackage{hyperref}
\usepackage[usenames,dvipsnames,svgnames,table,rgb]{xcolor}
\hypersetup{				% Гиперссылки
	unicode=true,           % русские буквы в раздела PDF
	pdftitle={Заголовок},   % Заголовок
	pdfauthor={Автор},      % Автор
	pdfsubject={Тема},      % Тема
	pdfcreator={Создатель}, % Создатель
	pdfproducer={Производитель}, % Производитель
	pdfkeywords={keyword1} {key2} {key3}, % Ключевые слова
	colorlinks=true,       	% false: ссылки в рамках; true: цветные ссылки
	linkcolor=violet,          % внутренние ссылки
	citecolor=black,        % на библиографию
	filecolor=orange,      % на файлы
	urlcolor= blue           % на URL
}

\usepackage{csquotes} % Еще инструменты для ссылок

%\usepackage[style=authoryear,maxcitenames=2,backend=biber,sorting=nty]{biblatex}

\usepackage{multicol} % Несколько колонок

\usepackage{tikz} % Работа с графикой
\usepackage{pgfplots}
\usepackage{pgfplotstable}

\renewcommand{\phi}{\varphi}
\renewcommand{\epsilon}{\varepsilon}

\title{Рецензия на работу Дорина Д. <<Моделирование показания fMRI по видео, показанному человеку>>}
\author{Сёмкин К.}
\date{}

\begin{document}
	\maketitle
	
	\section*{Аннотация}
	
	Имеется ясное и сжатое описание научной работы, можно добавить пару слов о результатах проведённого анализа.
	
	\section*{Введение}
	
	Введение хорошо описывает тематику работы и конкретные сущности исследования, все ссылки релевантны и уместны. Поставленная задача обозначена. Основные замечания:
	
	\begin{itemize}
		\item можно добавить информацию о других подходах в обработке видеорядов
		\item информацию о результатах эксперимента можно вывести в заключение, описание выборки также можно исключить из введения, оно и так имеется в подходящей секции
	\end{itemize}

	\section*{Постановка задачи}
	
	\begin{itemize}
		\item какая именно есть связь между $\mathcal{S}_0$ и $\mathcal{S}$ ?
		\item индексы у элементов $\mathcal{S}$ и $\mathcal{P}$ можно сделать одинаковыми снизу
	\end{itemize}

	\section*{Описание модели}
	
	Обозначения читаемы и понятны, модель поставлена корректно. Процесс получения решения, а также его постобработка и метрики качества обозначены. Основные замечания:
	
	\begin{itemize}
		\item правильно ли понимаю, что число параметров $d \times X_s \times Y_s \times Z_s$ ? Насколько вычислительно сложна реальная модель? :) 
		\item кажется, что выбранный вид функции потерь (покомпонентный) имеет некоторое необозначенное предположение о связи параметров модели $w_{ijk}$. Потери без доп. предположений выглядели бы как сумма квадратов ошибок по всему тензору, т.е. по всем индексам $i, j, k$ ?
	\end{itemize}

	\section*{Эксперимент}
	
	\begin{itemize}
		\item можно поподробней описать демонстрацию работы алгоритма, т.е. для какого агента и в какой момент времени получен данный срез
	\end{itemize}

	\section*{Анализ ошибки}
	
	Всё отлично, графики читаемые и ясно отображают закономерности в зависимости ошибок от гиперпараметров. Все графики содержат достаточные описания. 
	
	\begin{itemize}
		\item интересно, что даже при задержке в 100 с MSE растёт всего лишь на $25e-6$ 
		\item про раздел 5.2: я не эксперт в статике, но на рис. 4 разве не продемонстрирована ЦПТ?)
		\item интересный результат о преемственности весов между разными агентами
		\item можно подробней обсудить неустойчивость модели к шуму, обосновать какой-нибудь математикой, что именно значит этот результат? 
	\end{itemize}

	\section*{Заключение}
	
	Можно написать о предполагаемых улучшениях и доработках модели. Также можно обсудить возрастание сложности модели при возрастании размерностей снимков фМРТ.
	
\end{document}