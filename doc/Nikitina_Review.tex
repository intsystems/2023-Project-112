\documentclass[12pt, a4paper]{extarticle}

\usepackage{arxiv}

\usepackage[T2A]{fontenc}
\usepackage[utf8]{inputenc}
\usepackage[english, russian]{babel}
% \usepackage{cmap}
\usepackage{url}
\usepackage{booktabs}
\usepackage{nicefrac}
\usepackage{microtype}
\usepackage{lipsum}
\usepackage{graphicx}
\usepackage{epstopdf}
\usepackage{subfig}
\usepackage[square,sort,comma,numbers]{natbib}
\usepackage{doi}
\usepackage{multicol}
\usepackage{multirow}
\usepackage{tabularx}
\usepackage{float}

\usepackage{tikz}
\usetikzlibrary{matrix}

% Algorithms
\usepackage{algpseudocode}
\usepackage{algorithm}

%% Шрифты
\usepackage{euscript} % Шрифт Евклид
\usepackage{mathrsfs} % Красивый матшрифт
\usepackage{extsizes} % Возможность сделать 14-й шрифт
\usepackage{bm}

\usepackage{makecell} % diaghead in a table
\usepackage{amsmath,amsfonts,amssymb,amsthm,mathtools,dsfont}
\usepackage{icomma}
\usepackage[labelfont=bf]{caption}
\usepackage{subfig} % for subfigures
\usepackage{wrapfig}

\newcommand{\bz}{\mathbf{z}}
\newcommand{\bx}{\mathbf{x}}
\newcommand{\by}{\mathbf{y}}
\newcommand{\bv}{\mathbf{v}}
\newcommand{\bw}{\mathbf{w}}
\newcommand{\ba}{\mathbf{a}}
\newcommand{\bb}{\mathbf{b}}
\newcommand{\bp}{\mathbf{p}}
\newcommand{\bq}{\mathbf{q}}
\newcommand{\bt}{\mathbf{t}}
\newcommand{\bu}{\mathbf{u}}
\newcommand{\bs}{\mathbf{s}}
\newcommand{\bT}{\mathbf{T}}
\newcommand{\bX}{\mathbf{X}}
\newcommand{\bZ}{\mathbf{Z}}
\newcommand{\bS}{\mathbf{S}}
\newcommand{\bH}{\mathbf{H}}
\newcommand{\bW}{\mathbf{W}}
\newcommand{\bY}{\mathbf{Y}}
\newcommand{\bU}{\mathbf{U}}
\newcommand{\bQ}{\mathbf{Q}}
\newcommand{\bP}{\mathbf{P}}
\newcommand{\bA}{\mathbf{A}}
\newcommand{\bB}{\mathbf{B}}
\newcommand{\bC}{\mathbf{C}}
\newcommand{\bE}{\mathbf{E}}
\newcommand{\bF}{\mathbf{F}}
\newcommand{\bomega}{\boldsymbol{\omega}}
\newcommand{\btheta}{\boldsymbol{\theta}}
\newcommand{\bgamma}{\boldsymbol{\gamma}}
\newcommand{\bdelta}{\boldsymbol{\delta}}
\newcommand{\bPsi}{\boldsymbol{\Psi}}
\newcommand{\bpsi}{\boldsymbol{\psi}}
\newcommand{\bxi}{\boldsymbol{\xi}}
\newcommand{\bchi}{\boldsymbol{\chi}}
\newcommand{\bzeta}{\boldsymbol{\zeta}}
\newcommand{\blambda}{\boldsymbol{\lambda}}
\newcommand{\beps}{\boldsymbol{\varepsilon}}
\newcommand{\bZeta}{\boldsymbol{Z}}
% mathcal
\newcommand{\cX}{\mathcal{X}}
\newcommand{\cY}{\mathcal{Y}}
\newcommand{\cW}{\mathcal{W}}

\newcommand{\dH}{\mathds{H}}
\newcommand{\dR}{\mathds{R}}
% transpose
\newcommand{\T}{^{\mathsf{T}}}

% \renewcommand{\shorttitle}{\textit{arXiv} Шаблон}
\renewcommand{\epsilon}{\ensuremath{\varepsilon}}
\renewcommand{\phi}{\ensuremath{\varphi}}
\renewcommand{\kappa}{\ensuremath{\varkappa}}
\renewcommand{\le}{\ensuremath{\leqslant}}
\renewcommand{\leq}{\ensuremath{\leqslant}}
\renewcommand{\ge}{\ensuremath{\geqslant}}
\renewcommand{\geq}{\ensuremath{\geqslant}}
\renewcommand{\emptyset}{\varnothing}

\DeclareMathOperator*{\argmax}{arg\,max}  % in your preamble
\DeclareMathOperator*{\argmin}{arg\,min}  % in your preamble 

\usepackage{hyperref}
% \usepackage[usenames,dvipsnames,svgnames,table,rgb]{xcolor}

\hypersetup{
	unicode=true,
	colorlinks=true,
	linkcolor=black,        % внутренние ссылки
	citecolor=blue,         % на библиографию
	filecolor=magenta,      % на файлы
	urlcolor=blue           % на URL
}

\graphicspath{{./figures}}

\usepackage{enumitem} % Для модификаций перечневых окружений

\theoremstyle{definition} % "Определение"
\newtheorem{definition}{Опр.}[section]

\usepackage{etoolbox}

\makeatletter
\expandafter\patchcmd\csname\string\algorithmic\endcsname{\itemsep\z@}{\itemsep=1.5mm}{}{}
\makeatother

\newcommand{\myfigref}[2]{~\ref{#1}.\subref{#2}}% <---- a new macro for referring to a subfigure

% убираем номера страниц
\pagenumbering{gobble}

\begin{document} % конец преамбулы, начало документа

\begin{center}
    \textsf{\textbf{
        Рецензия на рукопись\\
        <<Автоматическое выделение терминов для тематического моделирования>>\\
        Никитина~М.\,А.
    }}
\end{center}

\begin{center}
    \textbf{1. Новизна, актуальность и обоснованность работы}
\end{center}

Четко сформулирована цель работы. Понятна и ясна ее актуальность.
Указано, в чем заключается новизна. Тем не менее, стоит привести дополнительные ссылки
на предшествующие исследования данной проблемы.

\begin{center}
    \textbf{2. Список ошибок, недочетов и замечаний}
\end{center}

\subsubsection*{Аннотация}

Раздел написан качественно и понятно. 
Опечатка: уз\textbf{у}коспециализированных $\to$ узкоспециализированных.

\subsubsection*{Введение}

Абзац~4, предложение <<Нейросеть использована готовая>> стоит пояснить подробнее или заменить.
Абзац~6, предложение <<Предшествующие исследования предлагаемого подхода~\ldots>> стоит
дополнить ссылкой на соответствующие работы.
Здесь и далее вместо короткого тире~\verb|--| нужно использовать неразрывное длинное~\verb|~---|.
Опечатка: эмбе\textbf{д}инг $\to$ эмбеддинг.

\subsubsection*{Постановка задачи}

Задача поставлена четко, ясна проблема и мотивация.

\subsubsection*{Вычислительный эксперимент}

Абзац~2, предложение <<Для проведения первого эксперимента из статей удаляются числа, заголовки
и ссылки на литературу~\ldots>>, стоит пояснить, вручную это делается, или же написан скрипт.
Рассматривается поиск терминов из одного слова. Далее проводится анализ всех терминов, но 
непонятно, есть ли ограничение по количеству слов в них.
Абзац~3, предложение <<Так как датасет состоит из статей, написанных на английском языке~\ldots>>,
не до конца понятно, как влияет язык текста на использование стемминга.
Для ссылок на таблицы и графики стоит использовать неразрывный пробел:~\verb|График~\ref{label}|.

\subsubsection*{Теоретическая часть}

Теоретическое описание метода должно идти до вычислительного эксперимента.
Опечатки: под формулой (8) исправить $\alpha td \to \alpha_{td}$, в формуле (9) также исправить
$\varphi \omega s \to \varphi_{\omega s}$.  

\subsubsection*{Анализ ошибки}

В Таблице~2 во всех столбцах кроме <<Порог>> используются точки в качестве разделителей для 
десятичных дробей, поэтому и в данном столбце следует использовать их.
В Таблицах~2 и 3 в пределах одного столбца приведены числа с разным количеством цифр после разделяющей
точки, лучше добавить нули в конце, например: $0.1 \to 0.100$.
Абзац~2, стоит заменить <<от худшего до лучшего>> $\to$ <<от худшего к лучшему>>.
Опечатка: в Абзаце~1 исправить Rec\textbf{q}ll $\to$ Recall.

\subsubsection*{Заключение}

Работа находится в стадии завершения, поэтому можно ожидать скорого написания данного раздела.

\begin{center}
    \textbf{3. Комментарии к коду}
\end{center}

Код написан аккуратно, четко и понятно. Приведены подробные комментарии к функциям и конкретным
фрагментам кода. Легко интерпретируется и расширяется.

\begin{center}
    \textbf{4. Общее мнение о работе}
\end{center}

Работа выполнена качественно и удовлетворяет всем требованиям, предъявляемым к статье.
Читается легко, все необходимые термины введены по ходу повествования.

\vspace{2cm}

\begin{flushleft}
    Рецензент:\\
    Киселев~Н.\,С.
\end{flushleft}

\end{document}