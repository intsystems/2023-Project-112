\documentclass[12pt,twoside]{article}
\usepackage{jmlda}
\usepackage[square,sort,comma,numbers]{natbib}
%\NOREVIEWERNOTES
\title
    [Моделирование показания FMRI по видео, показанному человеку] % Краткое название; не нужно, если полное название влезает в~колонтитул
    {Моделирование показания FMRI по видео, показанному человеку.}
\author
    [Дорин~Д.\,Д.] % список авторов для колонтитула; не нужен, если основной список влезает в колонтитул
    {Дорин~Д.\,Д., Киселев~Н.\,С., Грабовой~А.\,В.} % основной список авторов, выводимый в оглавление
    [Дорин~Д.\,Д.$^{1,2}$, Киселев~Н.\,С.$^{1,2}$, Грабовой~А.\,В.$^2$] % список авторов, выводимый в заголовок; не нужен, если он не отличается от основного

\email
    {dorin.dd@phystech.edu}
\organization
    {$^1$Организация; $^2$Организация}
\abstract
    {Внутричерепные записи человека являются ценным и редким источником информации о мозге. 
    Одним из методов исследования головного мозга является FMRI (Функциональная магнитно-резонансная томография). 
    В данной работе исследуется задача прогнозирования показаний датчиков FMRI по видеоряду, показанному человеку. 
    Предложен метод апроксимации показаний FMRI по видеоряду на основе трансформер моделей. 
    Подтверждена зависимость между показаниями датчиков и восприятием внешнего мира человеком. 

\bigskip
\textbf{Ключевые слова}: \emph {FMRI, видеоряд, трансформер модель}.}
\titleEng
    {JMLDA paper example: file jmlda-example.tex}
\authorEng
    {Author~F.\,S.$^1$, CoAuthor~F.\,S.$^2$, Name~F.\,S.$^2$}
\organizationEng
    {$^1$Organization; $^2$Organization}
\abstractEng
    {This document is an example of paper prepared with \LaTeXe\
    typesetting system and style file \texttt{jmlda.sty}.

    \bigskip
    \textbf{Keywords}: \emph{keyword, keyword, more keywords}.}
\begin{document}
\maketitle
%\linenumbers
\section{Введение}
Несмотря на множество достижений современной науки, человеческий мозг остается одним из самых загадочных объектов. 
Поэтому задача исследования человеческого мозга актуальна в наши дни.

В работе исследуется зависимость между показанием FMRI и видеорядом, просмотренным человеком.
Функциональная магнитно-резонансная томография (FMRI, Functional magnetic resonance imaging) — разновидность 
магнитно-резонансной томографии, которая проводится с целью измерения гемодинамических реакций (изменений в токе крови), 
вызванных нейронной активностью головного или спинного мозга.
Этот метод основывается на том, что мозговой кровоток и активность нейронов связаны между собой.
Когда область мозга активна, приток крови к этой области также увеличивается.
FMRI позволяет определить активацию определенной области головного мозга во время нормального 
его функционирования под влиянием различных физических факторов (например, движение тела или просмотра видеоряда).
Исследование основано на обширном датасете, представленном в работе Юлии Березуцкой \citep{Berezutskaya2022}.
Набор данных состоит из видеорядов и FMRI снимков.

\section{Постановка задачи}

\section{Вычислительный эксперимент}


\section{Анализ ошибки}

\section{Заключение}

\bibliographystyle{plain}

\bibliography{dorin.bib}

%\begin{thebibliography}{1}

%\bibitem{author09anyscience}
%    \BibAuthor{Author\;N.}
%    \BibTitle{Paper title}~//
%    \BibJournal{10-th Int'l. Conf. on Anyscience}, 2009.  Vol.\,11, No.\,1.  Pp.\,111--122.
%\bibitem{myHandbook}
%    \BibAuthor{Автор\;И.\,О.}
%    Название книги.
%    Город: Издательство, 2009. 314~с.
%\bibitem{author09first-word-of-the-title}
%    \BibAuthor{Автор\;И.\,О.}
%    \BibTitle{Название статьи}~//
%    \BibJournal{Название конференции или сборника},
%    Город:~Изд-во, 2009.  С.\,5--6.
%\bibitem{author-and-co2007}
%    \BibAuthor{Автор\;И.\,О., Соавтор\;И.\,О.}
%    \BibTitle{Название статьи}~//
%    \BibJournal{Название журнала}. 2007. Т.\,38, \No\,5. С.\,54--62.
%\bibitem{bibUsefulUrl}
%    \BibUrl{www.site.ru}~---
%    Название сайта.  2007.
%\bibitem{voron06latex}
%    \BibAuthor{Воронцов~К.\,В.}
%    \LaTeXe\ в~примерах.
%    2006.
%    \BibUrl{http://www.ccas.ru/voron/latex.html}.
%\bibitem{Lvovsky03}
%    \BibAuthor{Львовский~С.\,М.} Набор и вёрстка в пакете~\LaTeX.
%    3-е издание.
%    Москва:~МЦHМО, 2003.  448~с.
%\end{thebibliography}

% Решение Программного Комитета:
%\ACCEPTNOTE
%\AMENDNOTE
%\REJECTNOTE
\end{document}
