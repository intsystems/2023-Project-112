\documentclass[12pt,twoside]{article}
\usepackage{jmlda}
\usepackage[square,sort,comma,numbers]{natbib}
\newcommand{\cyrchar}[1]{\foreignlanguage{russian}{#1}}
%\NOREVIEWERNOTES
\title
    [Моделирование показания фМРТ по видео, показанному человеку] % Краткое название; не нужно, если полное название влезает в~колонтитул
    {Моделирование показания фМРТ по видео, показанному человеку.}
\author
    [Дорин~Д.\,Д.] % список авторов для колонтитула; не нужен, если основной список влезает в колонтитул
    {Дорин~Д.\,Д., Киселев~Н.\,С., Грабовой~А.\,В.} % основной список авторов, выводимый в оглавление
    [Дорин~Д.\,Д.$^{1,2}$, Киселев~Н.\,С.$^{1,2}$, Грабовой~А.\,В.$^2$] % список авторов, выводимый в заголовок; не нужен, если он не отличается от основного

\email
    {dorin.dd@phystech.edu}
\organization
    {$^1$Организация; $^2$Организация}
\abstract
    {В данной работе исследуется задача прогнозирования показаний датчиков фМРТ по видеоряду, 
    показанному человеку. 
    Предложен метод апроксимации показаний фМРТ по видеоряду на основе Трансформер моделей. 
    Проанализирована зависимость между показаниями датчиков и восприятием внешнего мира человеком.
    Эффективность предложенного подхода демонстрируется на наборе данных, 
    собранных у большой группы людей, когда они смотрели короткий аудиовизуальный фильм.

\bigskip
\textbf{Ключевые слова}: \emph {фМРТ, видеоряд, Трансформер модель}.}
\titleEng
    {JMLDA paper example: file jmlda-example.tex}
\authorEng
    {Author~F.\,S.$^1$, CoAuthor~F.\,S.$^2$, Name~F.\,S.$^2$}
\organizationEng
    {$^1$Organization; $^2$Organization}
\abstractEng
    {This document is an example of paper prepared with \LaTeXe\
    typesetting system and style file \texttt{jmlda.sty}.
    
\bigskip
\textbf{Keywords}: \emph{keyword, keyword, more keywords}.}

\begin{document}
\maketitle
%\linenumbers
\section{Введение}
Человеческий мозг один из самых интересных объектов исследования. 
Внутречерепные записи человека являются редким и ценным источником информации о мозге.
Поэтому исследование методов получения данных о функциональной активности коры 
головного мозга актуально в наши дни.

Одним из методов исследования активности головного мозга является функциональная магнитно-резонансная томография.
фМРТ~--- разновидность магнитно-резонансной томографии, которая проводится с целью измерения гемодинамических реакций~--- 
изменений в потоке крови, вызванных нейронной активностью головного или спинного мозга.
Этот метод основывается на связи мозгового кровотока и активности нейронов. Когда область мозга активна, 
приток крови к этой области также увеличивается. 
фМРТ позволяет определить активацию определенной области головного мозга во время нормального функционирования под 
влиянием различных заданий, например, зрительных, когнитивных,  моторных,  речевых.
В работе Алены Беляевской \citep{Belyaevskaya2018} собраны современные возможности фМРТ в нейровизуализации~--- 
общее название нескольких методов, позволяющих визуализировать структуру, функции и биохимические характеристики мозга.

В работе Юлии Березуцкой \citep{Berezutskaya2022} собран обширный набор данных, сотоящий из видеорядов, просмотренных 
человеком, и соответствующих снимков фМРТ. Одна из проблем при работе с данными нейровизуализации~--- шум, вызванный 
дивжением головы, биением сердца, тепловыми эффектами и др. 
В работе Максима Шараева \citep{https://doi.org/10.48550/arxiv.1804.10167} рассмотрены подходы к подготовке,
предварительной обработке, шумоподавлению, направленные на устранение артефактов, вредных 
для распознавания образов, а также методы классификации данных нейровизуализации.

Наиболее известные методы обработки видео основаны на 3D свертках. 
3D в отличие от 2D сверток одновременно работают с пространственной  
и временной частью информации. Существенный недостаток данных методов~--- 
сильное увеличение числа параметров модели и большие вычислительные затраты.
В работе используется более современная архитектура~--- Трансформер модель.
Впервые модель Трансформер была предложена в статье <<Attention Is All You Need>> Ashish Vaswani 
\citep{https://doi.org/10.48550/arxiv.1706.03762}. Архитектура активно применяется в области машинного перевода.
А в 2022 году появилась работа Shen Yan \citep{transformer} на тему адаптации архитектуры Трансформер для работы с видеорядами. 
Данная архитектура учитывает пространственно-временные зависимости и повышает скорость обучения засчет attention слоев.
Сама модель состоит из кодирующего компонента, декодирующего компонента и связи между ними. Каждый компонент состоит из стека 
энкодеров и декодеров соотвественно. 
Входящая последовательность, поступающая в энкодер, сначала проходит через attention слой, помогающий энкодеру 
посмотреть на другие слова во входящем объеме во время кодирования конкретного элемента. 
Выход attention слоя отправляется в нейронную сеть прямого распространения. 
Аналогично устроен декодер, за исключением наличия еще одного слоя внимания, помогающего фокусироваться на релевантных элементах.

В данной работе предлагается метод аппроксимации показаний датчиков фМРТ по видеоряду.
Полученная в ходе экспериментов корреляционная картина между  фМРТ снимками и просмотренными видеорядами подтверждает 
зависимость между показаниями фМРТ и восприятием внешнего мира человеком. 

Проверка метода проводится на выборке, представленной в работе \citep{Berezutskaya2022}. 
Набор данных включает в себя в себя записи фМРТ 30 участников в возрасте от 7 до 47 лет во время 
выполнения одинаковой задачи и записи внутричерепной электроэнцефалографии 51 участникa в возрасте от 5 до 55 лет. 


\section{Постановка задачи}
Пусть $\Omega$~--- видеоряд:
\begin{equation}
    \Omega = (\omega_{1}, \dots, \omega_{n}),
\end{equation}
где $\omega_i \in \mathbb{R}^{W_{\omega} \times H_{\omega} \times C_{\omega}}$~--- изображение, $W_{\omega}$~---
ширина изображения, $H_{\omega}$~--- высота изображения и $C_{\omega}$~--- число каналов.
Введем также $\mathcal{S}$~--- последовательность фМРТ снимков:
\begin{equation}
    \mathcal{S} = (s_{1}, \dots, s_{m}),
\end{equation}
где $s_i \in \mathbb{R}^{W_{s} \times H_{s} \times C_{s}}$~--- фМРТ снимок, $W_{s}$~---
ширина изображения, $H_{s}$~--- высота изображения и $C_{s}$~--- число каналов.

Необходимо построить отображение $f: \Omega \to \mathcal{S}$.

\section{Вычислительный эксперимент}


\section{Анализ ошибки}

\section{Заключение}



\bibliographystyle{plain}

\bibliography{dorin.bib}

%\begin{thebibliography}{1}

%\bibitem{author09anyscience}
%    \BibAuthor{Author\;N.}
%    \BibTitle{Paper title}~//
%    \BibJournal{10-th Int'l. Conf. on Anyscience}, 2009.  Vol.\,11, No.\,1.  Pp.\,111--122.
%\bibitem{myHandbook}
%    \BibAuthor{Автор\;И.\,О.}
%    Название книги.
%    Город: Издательство, 2009. 314~с.
%\bibitem{author09first-word-of-the-title}
%    \BibAuthor{Автор\;И.\,О.}
%    \BibTitle{Название статьи}~//
%    \BibJournal{Название конференции или сборника},
%    Город:~Изд-во, 2009.  С.\,5--6.
%\bibitem{author-and-co2007}
%    \BibAuthor{Автор\;И.\,О., Соавтор\;И.\,О.}
%    \BibTitle{Название статьи}~//
%    \BibJournal{Название журнала}. 2007. Т.\,38, \No\,5. С.\,54--62.
%\bibitem{bibUsefulUrl}
%    \BibUrl{www.site.ru}~---
%    Название сайта.  2007.
%\bibitem{voron06latex}
%    \BibAuthor{Воронцов~К.\,В.}
%    \LaTeXe\ в~примерах.
%    2006.
%    \BibUrl{http://www.ccas.ru/voron/latex.html}.
%\bibitem{Lvovsky03}
%    \BibAuthor{Львовский~С.\,М.} Набор и вёрстка в пакете~\LaTeX.
%    3-е издание.
%    Москва:~МЦHМО, 2003.  448~с.
%\end{thebibliography}

% Решение Программного Комитета:
%\ACCEPTNOTE
%\AMENDNOTE
%\REJECTNOTE
\end{document}
