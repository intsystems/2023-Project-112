\documentclass[12pt,twoside]{article}
\usepackage{jmlda}
%\NOREVIEWERNOTES
\title
    [Моделирование показания FMRI по видео, показанному человеку] % Краткое название; не нужно, если полное название влезает в~колонтитул
    {Моделирование показания FMRI по видео, показанному человеку.}
\author
    [Дорин~Д.\,Д.] % список авторов для колонтитула; не нужен, если основной список влезает в колонтитул
    {Дорин~Д.\,Д., Киселев~Н.\,С., Грабовой~А.\,В.} % основной список авторов, выводимый в оглавление
    [Дорин~Д.\,Д.$^{1,2}$, Киселев~Н.\,С.$^{1,2}$, Грабовой~А.\,В.$^2$] % список авторов, выводимый в заголовок; не нужен, если он не отличается от основного

\email
    {dorin.dd@phystech.edu}
\organization
    {$^1$Организация; $^2$Организация}
\abstract
    {Внутричерепные записи человека являются ценным и редким источником информации о мозге. Одним из методов исследования головного мозга является фМРТ (Функциональная магнитно-резонансная томография). В данной работе исследуется задача прогнозирования показаний датчиков фМРТ по видеоряду, показанному человеку. Предложен метод апроксимации показаний фМРТ по видеоряду на основе трансформер моделей. Подтверждена зависимость между показаниями датчиков и восприятием внешнего мира человеком. 

\bigskip
\textbf{Ключевые слова}: \emph {ключевое слово, ключевое слово,
еще ключевые слова}.}
\titleEng
    {JMLDA paper example: file jmlda-example.tex}
\authorEng
    {Author~F.\,S.$^1$, CoAuthor~F.\,S.$^2$, Name~F.\,S.$^2$}
\organizationEng
    {$^1$Organization; $^2$Organization}
\abstractEng
    {This document is an example of paper prepared with \LaTeXe\
    typesetting system and style file \texttt{jmlda.sty}.

    \bigskip
    \textbf{Keywords}: \emph{keyword, keyword, more keywords}.}
\begin{document}
\maketitle
%\linenumbers
\section{Введение}
новизна работы заключается в отсутствии работ ...
После аннотации, но перед первым разделом,
располагается введение, включающее в себя
описание предметной области,
обоснование актуальности задачи,
краткий обзор известных результатов,
и~т.\,п~\cite{author09anyscience,myHandbook,author09first-word-of-the-title,voron06latex,author-and-co2007,Lvovsky03}.

\section{Постановка задачи}

\section{Вычислительный эксперимент}


\section{Анализ ошибки}

\section{Заключение}

\bibliographystyle{unsrt}
\bibliography{jmlda-bib}
%\begin{thebibliography}{1}

%\bibitem{author09anyscience}
%    \BibAuthor{Author\;N.}
%    \BibTitle{Paper title}~//
%    \BibJournal{10-th Int'l. Conf. on Anyscience}, 2009.  Vol.\,11, No.\,1.  Pp.\,111--122.
%\bibitem{myHandbook}
%    \BibAuthor{Автор\;И.\,О.}
%    Название книги.
%    Город: Издательство, 2009. 314~с.
%\bibitem{author09first-word-of-the-title}
%    \BibAuthor{Автор\;И.\,О.}
%    \BibTitle{Название статьи}~//
%    \BibJournal{Название конференции или сборника},
%    Город:~Изд-во, 2009.  С.\,5--6.
%\bibitem{author-and-co2007}
%    \BibAuthor{Автор\;И.\,О., Соавтор\;И.\,О.}
%    \BibTitle{Название статьи}~//
%    \BibJournal{Название журнала}. 2007. Т.\,38, \No\,5. С.\,54--62.
%\bibitem{bibUsefulUrl}
%    \BibUrl{www.site.ru}~---
%    Название сайта.  2007.
%\bibitem{voron06latex}
%    \BibAuthor{Воронцов~К.\,В.}
%    \LaTeXe\ в~примерах.
%    2006.
%    \BibUrl{http://www.ccas.ru/voron/latex.html}.
%\bibitem{Lvovsky03}
%    \BibAuthor{Львовский~С.\,М.} Набор и вёрстка в пакете~\LaTeX.
%    3-е издание.
%    Москва:~МЦHМО, 2003.  448~с.
%\end{thebibliography}

% Решение Программного Комитета:
%\ACCEPTNOTE
%\AMENDNOTE
%\REJECTNOTE
\end{document}
