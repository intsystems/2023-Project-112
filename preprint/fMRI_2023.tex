\documentclass[a4paper, 12pt]{extarticle}

\usepackage{arxiv}

\usepackage[T1]{fontenc}
\usepackage[utf8]{inputenc}
%\usepackage[english, russian]{babel}
\usepackage{lipsum}         % Can be removed after putting your text content
\usepackage{url}
\usepackage{booktabs}
\usepackage{nicefrac}
\usepackage{microtype}
\usepackage{lipsum}
\usepackage{graphicx}
\usepackage{epstopdf}
\usepackage{subfig}
\usepackage[sort,comma,numbers,]{natbib}
\usepackage{doi}
\usepackage{multicol}
\usepackage{multirow}
\usepackage{tabularx}
\usepackage{float}

\usepackage{tikz}
\usetikzlibrary{matrix}

% Algorithms
\usepackage{algpseudocode}
\usepackage{algorithm}

%% Шрифты
\usepackage{euscript} % Шрифт Евклид
\usepackage{mathrsfs} % Красивый матшрифт
%\usepackage{extsizes} % Возможность сделать 14-й шрифт
\usepackage{bm}

\usepackage{makecell} % diaghead in a table
\usepackage{amsmath,amsfonts,amssymb,amsthm,mathtools,dsfont}
\usepackage{icomma}
\usepackage[labelfont=bf]{caption}
\usepackage{subfig} % for subfigures
\usepackage{wrapfig}

\newcommand{\bz}{\mathbf{z}}
\newcommand{\bx}{\mathbf{x}}
\newcommand{\by}{\mathbf{y}}
\newcommand{\bv}{\mathbf{v}}
\newcommand{\bw}{\mathbf{w}}
\newcommand{\ba}{\mathbf{a}}
\newcommand{\bb}{\mathbf{b}}
\newcommand{\bp}{\mathbf{p}}
\newcommand{\bq}{\mathbf{q}}
\newcommand{\bt}{\mathbf{t}}
\newcommand{\bu}{\mathbf{u}}
\newcommand{\bs}{\mathbf{s}}
\newcommand{\bT}{\mathbf{T}}
\newcommand{\bX}{\mathbf{X}}
\newcommand{\bZ}{\mathbf{Z}}
\newcommand{\bS}{\mathbf{S}}
\newcommand{\bH}{\mathbf{H}}
\newcommand{\bW}{\mathbf{W}}
\newcommand{\bY}{\mathbf{Y}}
\newcommand{\bU}{\mathbf{U}}
\newcommand{\bQ}{\mathbf{Q}}
\newcommand{\bP}{\mathbf{P}}
\newcommand{\bA}{\mathbf{A}}
\newcommand{\bB}{\mathbf{B}}
\newcommand{\bC}{\mathbf{C}}
\newcommand{\bE}{\mathbf{E}}
\newcommand{\bF}{\mathbf{F}}
\newcommand{\bomega}{\boldsymbol{\omega}}
\newcommand{\btheta}{\boldsymbol{\theta}}
\newcommand{\bgamma}{\boldsymbol{\gamma}}
\newcommand{\bdelta}{\boldsymbol{\delta}}
\newcommand{\bPsi}{\boldsymbol{\Psi}}
\newcommand{\bpsi}{\boldsymbol{\psi}}
\newcommand{\bxi}{\boldsymbol{\xi}}
\newcommand{\bchi}{\boldsymbol{\chi}}
\newcommand{\bzeta}{\boldsymbol{\zeta}}
\newcommand{\blambda}{\boldsymbol{\lambda}}
\newcommand{\beps}{\boldsymbol{\varepsilon}}
\newcommand{\bZeta}{\boldsymbol{Z}}
% mathcal
\newcommand{\cX}{\mathcal{X}}
\newcommand{\cY}{\mathcal{Y}}
\newcommand{\cW}{\mathcal{W}}

\newcommand{\dH}{\mathds{H}}
\newcommand{\dR}{\mathds{R}}
% transpose
\newcommand{\T}{^{\mathsf{T}}}

% \renewcommand{\shorttitle}{\textit{arXiv} Шаблон}
\renewcommand{\epsilon}{\ensuremath{\varepsilon}}
\renewcommand{\phi}{\ensuremath{\varphi}}
\renewcommand{\kappa}{\ensuremath{\varkappa}}
\renewcommand{\le}{\ensuremath{\leqslant}}
\renewcommand{\leq}{\ensuremath{\leqslant}}
\renewcommand{\ge}{\ensuremath{\geqslant}}
\renewcommand{\geq}{\ensuremath{\geqslant}}
\renewcommand{\emptyset}{\varnothing}

\DeclareMathOperator*{\argmax}{arg\,max}  % in your preamble
\DeclareMathOperator*{\argmin}{arg\,min}  % in your preamble 

\usepackage{hyperref}
%\usepackage[usenames,dvipsnames,svgnames,table,rgb]{xcolor}

\hypersetup{
	unicode=true,
	colorlinks=true,
	linkcolor=red,
	anchorcolor=black,
	citecolor=blue,
	filecolor=cyan,
	menucolor=red,
	runcolor=cyan,
	urlcolor=magenta
}

\graphicspath{{./figures}}

\usepackage{enumitem} % Для модификаций перечневых окружений
\usepackage{etoolbox}

\makeatletter
\expandafter\patchcmd\csname\string\algorithmic\endcsname{\itemsep\z@}{\itemsep=1.5mm}{}{}
\makeatother

\newcommand{\myfigref}[2]{~\ref{#1}.\subref{#2}}% <---- a new macro for referring to a subfigure

% Here you can change the date presented in the paper title
%\date{September 9, 1985}
% Or remove it
%\date{}

\newif\ifuniqueAffiliation
% Comment to use multiple affiliations variant of author block 
\uniqueAffiliationtrue

\ifuniqueAffiliation % Standard variant of author block
\author{
	Dorin Daniil \\
	\texttt{dorin.dd@phystech.edu} \\
	\And
	Kiselev Nikita \\
	\texttt{kiselev.ns@phystech.edu} \\
	\And
	Grabovoy Andrey \\
	\texttt{grabovoy.av@phystech.edu} \\
}
\else
% Multiple affiliations variant of author block
\usepackage{authblk}
\renewcommand\Authfont{\bfseries}
\setlength{\affilsep}{0em}
% box is needed for correct spacing with authblk
\newbox{\orcid}\sbox{\orcid}{\includegraphics[scale=0.06]{orcid.pdf}} 
\author[1]{%
	\href{https://orcid.org/0000-0000-0000-0000}{\usebox{\orcid}\hspace{1mm}David S.~Hippocampus\thanks{\texttt{hippo@cs.cranberry-lemon.edu}}}%
}
\author[1,2]{%
	\href{https://orcid.org/0000-0000-0000-0000}{\usebox{\orcid}\hspace{1mm}Elias D.~Striatum\thanks{\texttt{stariate@ee.mount-sheikh.edu}}}%
}
\affil[1]{Department of Computer Science, Cranberry-Lemon University, Pittsburgh, PA 15213}
\affil[2]{Department of Electrical Engineering, Mount-Sheikh University, Santa Narimana, Levand}
\fi

% Uncomment to override  the `A preprint' in the header
%\renewcommand{\headeright}{Technical Report}
%\renewcommand{\undertitle}{Technical Report}
\renewcommand{\shorttitle}{Forecasting fMRI Images From Video Sequences: Linear Model Analysis}

\title{Forecasting fMRI Images From Video Sequences: Linear Model Analysis}
\renewcommand{\abstractname}{Аннотация}

\title{Восстановление снимков фМРТ по просматриваемому видеоряду}

\author{
	Дорин Даниил \\
	\texttt{dorin.dd@phystech.edu} \\
	\And
	Киселев Никита \\
	\texttt{kiselev.ns@phystech.edu} \\
	\And
	Грабовой Андрей \\
	\texttt{grabovoy.av@phystech.edu}
}
\date{\today}

\begin{document}
\maketitle

\begin{abstract}

	Исследуется проблема восстановления зависимости между показаниями датчиков фМРТ
	и восприятием внешнего мира человеком.
	Проводится анализ зависимости между последовательностью снимков фМРТ и видеорядом,
	просматриваемыx человеком.
	На основе исследования зависимости предлагается метод аппроксимации показаний фМРТ по
	просматриваемому видеоряду.
	Для анализа предложенного метода проводится вычислительный эксперимент на
	выборке, полученной при томографическом обследовании большого числа испытуемых.
	На экспериментальных данных проводится анализ зависимости 
	качества работы метода от времени гемодинамической ответной реакции
	зависимости уровня кислорода в крови.
	Проверяются гипотезы об инвариантности весов модели относительно человека
	и обобщающей способности построенного метода.

\end{abstract}


\keywords{нейровизуализация \and фМРТ \and видеоряд \and зависимость между данными
\and проверка гипотез \and линейная модель \and линейная регрессия}

\section{Введение}

Совокупность методов, визуализирующих структуру и функции человеческого мозга,
называется \textit{нейровизуализацией}. Методы нейровизуализации \citep{puras2014neurovisualization}, такие как ЭКГ, КТ, МРТ и фМРТ,
используются для изучения мозга, а также для обнаружения заболеваний и психических расстройств.

\textit{Функциональная магнитно-резонансная томография} или \textit{фМРТ} (англ.~\textit{fMRI})
является разновидностью магнитно-резонансной томографии и основана на изменениях в токе крови,
вызванных нейронной активностью мозга \citep{Glover2011}.
Эти изменения происходят не моментально, а с некоторой задержкой,
которая составляет 4--8 с \citep{Bandettini1992}.
Она возникает из-за того, что сосудистая система достаточно долго реагирует
на потребность мозга в глюкозе \citep{Ogawa1990, LEBIHAN1995231, Logothetis2003}. 

При получении снимков фМРТ используются последовательности
эхопланарных изображений (EPI) \citep{Connelly1993, Kwong1992, Ogawa1992}.
Обработка участков с изменяющейся интенсивностью сигнала в
зависимости от способа активации, вида артефактов и длительности
проводится с помощью специальных методов и программ 
\citep{Bandettini1992, BAUDENDISTEL1995701, COX1996162}.
Обработанные результаты оформляются в виде карт активации,
которые совмещаются с локализацией анатомических образований
головконого мозга.

Метод фМРТ играет большую роль в нейровизуализации, однако имеет ряд важных ограничений.
В работах \citep{menon1999spatial, logothetis2008we} рассматриваются
временное и пространственное разрешения фМРТ. Временное разрешение является существенным
недостатком данного метода. Другой недостаток фМРТ~--- неизбежно возникающие шумы,
связанные с движением объекта в сканере, сердцебиением и дыханием человека, тепловыми
флуктуациями самого прибора и т.\,д. В работе \citep{1804.10167} предлагаются методы
подавления вышеперечисленных шумов на основе графов и демонстрируется их эффективность в задаче
выявления эпилепсии и депрессии.

При проведении фМРТ испытуемому дают различные тест-задания и
применяют внешние раздражители, вызывающие активацию определенных
локальных участков головного мозга, ответственных за выполнение
соответствующих его функций.
Применяются различные тест-задания: движения пальцами и конечностями
\citep{Roux1998, Papke1999}, поиск изображения и рассмотрение
шахматной доски \citep{Engel1994, Schneider1994},
прослушивание неспецифичных шумов, единичных слов
или связного текста \citep{Binder1994, Dymarkowski1998}.
Причиной изменения активности человеского мозга во время фМРТ-обследования
также может служить просмотр видеоматериала \citep{decety1997brain},
что является предметом исследования настоящей работы. 

Наиболее известные методы обработки видео основаны на 3D свертках \citep{tran2015learning}.
Отличие 3D от 2D сверток заключается в одновременной работе с пространственной и временной частью
информации. Существенный недостаток данных методов — сильное увеличение числа параметров модели и
большие вычислительные затраты. Одной из наиболее современных и улучшаемых архитектур
нейронных сетей для обработки изображений является остаточная нейронная сеть ResNet~\citep{he2015deep}.
Она позволяет обучать глубокие нейронные сети (до 152 слоев) с высокой точностью,
преодолевая проблему затухания градиента, которая возникает при обучении глубоких сетей.

Настоящая работа посвящена восстановлению зависимости между снимками фМРТ и видеорядом.
Используется предположение, что такая зависимость существует.
Кроме того, предполагается, что между снимком и видеорядом есть постоянная задержка во времени
\citep{Logothetis2003}.
Проверяется зависимость снимка фМРТ от одного изображения и предыдущего снимка.
Время задержки выступает в качестве гиперпараметра модели.
На основе анализа зависимости предлагается метод аппроксимации показаний фМРТ по
просматриваемому видеоряду.

Согласно исследованию \citep{anderson2006}, при фМРТ-обследовании пациентов,
просматривающих видеоряд, активируется определенная корковая сеть. Она включает
в себя в том числе задние центральные и фронтальные области. Находится эта сеть
преимущественно в правом полушарии. В настоящей работе рассматривается подход,
использующий для анализа времени задержки именно эти части головного мозга.

Данные, на которых проводятся проверка гипотезы зависимости и демонстрация работы построенного
метода, представлены в работе \citep{Berezutskaya2022}. Этот набор данных был получен при
обследовании группы из 63 испытуемых. Тридцать из них проходили обследование фМРТ.
Им предлагалось выполнить одно и то же задание~--- просмотреть короткий аудиовизуальный фильм.
Для него в рассматриваемой работе были сгенерированы аннотации, содержащие в том числе информацию о времени появления и исчезновения
отдельных слов, объектов и персонажей. Методы аудио- и видеоаннотирования подробно излагаются в
\citep{boersma2018praat} и \citep{Berezutskaya2020}.

\section{Постановка задачи}

Задана частота кадров $\nu \in \mathbb{R}$ и продолжительность $t \in \mathbb{R}$ видеоряда.
Задан видеоряд
\begin{equation}
	\label{eq1}
	\bP = [\bp_1, \ldots, \bp_{\nu t}], \quad\
	\bp_{\ell} \in \mathbb{R}^{W \times H \times C},
\end{equation}
с шириной, высотой и числом каналов изображения $W, H$ и
$C$ соответственно.

Обозначим частоту снимков фМРТ $\mu \in \mathbb{R}$. Задана последовательность снимков
\begin{equation}
	\label{eq2}
	\bS = [\bs_1, \ldots, \bs_{\mu t}], \quad\
	\bs_{\ell} \in \mathbb{R}^{X \times Y \times Z},
\end{equation}
где $X, Y$ и $Z$~--- размерности воксельного изображения.

Задача состоит в построении отображения, которое бы учитывало задержку $\Delta t$ между
снимком фМРТ и видеорядом, а также предыдущие томографические показания. Формально, необходимо
найти такое отображение $\mathbf{g}$, что
\begin{equation}
	\label{eq3}
	\mathbf{g}(\bp_1, \ldots, \bp_{k_{\ell} - \nu \Delta t}; \bs_1, \ldots, \bs_{\ell-1}) = \bs_{\ell},
	\ \ell = 1, \ldots, \mu t,
\end{equation}
где для $\ell$-го снимка фМРТ номер соответствующего изображения $k_{\ell}$ определяется по формуле
\begin{equation}
	\label{eq4}
	k_{\ell} = \dfrac{\ell \cdot \nu}{\mu}.
\end{equation}

\section{Предлагаемый метод восстановления снимков фМРТ}

Обозначим снимок фМРТ как $\bs_{\ell} = [v^{\ell}_{ijk}] \in \mathbb{R}^{X \times Y \times Z}$,
где $v^{\ell}_{ijk} \in \mathbb{R}_+$~--- значение соответствующего вокселя.
Предположим, что для последовательности снимков выполнено марковское свойство,
т.е. каждый снимок зависит только от одного изображения и предыдущего снимка.
Тогда соответствующее отображение можно записать в виде
\begin{equation}
	\label{eq5}
	\mathbf{g}(\bp_{k_{\ell} - \nu \Delta t}) = \bs_{\ell} - \bs_{\ell-1} = \bdelta_{\ell}, \ \ell = 2, \ldots, \mu t.
\end{equation}
где $\bdelta_{\ell} = [v^{\ell}_{ijk} - v^{\ell-1}_{ijk}] = [\delta^{\ell}_{ijk}] \in \mathbb{R}^{X \times Y \times Z}$~--- разность между двумя последовательными снимками.

Отображение $\mathbf{g}: \mathbf{P} \to \mathbf{S}$ можно представить в виде композиции
двух других:
\[ \mathbf{g} = \bm{\varphi} \circ \bm{\psi}, \]
\vspace{-0.8cm}
\begin{align*}
	 & \bm{\psi}: \mathbf{P} \to \mathbb{R}^d
	\text{~--- векторизация изображения,}        \\
	 & \bm{\varphi}: \mathbb{R}^d \to \mathbf{S}
	\text{~--- восстанавливаемое отображение.}
\end{align*}

Для каждого изображения из видеоряда имеем вектор признакового описания размерности $d$:
\[ \bx_{\ell} = [x^{\ell}_1, \ldots, x^{\ell}_{d}]\T \in \mathbb{R}^{d}, \ {\ell} = 1, \ldots, \nu t. \]
Используется архитектура нейронной сети ResNet152 без последнего линейного слоя.

Учитывая \eqref{eq4}, суммарное число пар (изображение, снимок)
равно $N = \mu (t - \Delta t)$. Таким образом, для каждого вокселя задана выборка
\[ \mathfrak{D}_{ijk} = \{(\bx_{\ell}, \delta^{\ell}_{ijk}) \ | \ {\ell} = 2, \ldots, N \}. \]

Поставлена задача восстановления регрессии
\begin{equation}
	\label{eq6}
	y_{ijk}: \mathbb{R}^{d} \to \mathbb{R}.
\end{equation}

Используется линейная модель с вектором параметров
\[ \bw_{ijk} = [w^{ijk}_1, \ldots, w^{ijk}_{d}]\T \in \mathbb{R}^{d}: \]
\begin{equation}
	\label{eq7}
	f_{ijk}(\bx, \bw_{ijk}) = \langle \bx, \bw_{ijk} \rangle.
\end{equation}

Для модели $f_{ijk}$ с соответствующим ей вектором параметров $\bw_{ijk} \in \mathbb{R}^{d}$
определим квадратичную функцию потерь с $L_2$ регуляризацией:
\begin{equation}
	\label{eq8}
	\mathcal{L}_{ijk}(\bw_{ijk}) = \sum\limits_{\ell = 2}^{N} \big(f_{ijk}(\bx_{\ell}, \bw_{ijk}) - \delta^{\ell}_{ijk}\big)^2 + \alpha \| \bw_{ijk} \|_2^2,
\end{equation}
где $\alpha \in \mathbb{R}$~--- коэффициент регуляризации.

Требуется найти параметры, доставляющие минимум функционалу потерь $\mathcal{L}_{ijk}(\bw_{ijk})$
при заданных гиперпараметрах $\Delta t$ и $\alpha$:
\begin{equation}
	\label{eq9}
	\hat{\bw}_{ijk} = \argmin_{\bw_{ijk}} \mathcal{L}_{ijk}(\bw_{ijk}).
\end{equation}

Минимум функции потерь находится методом наименьших квадратов. Определим матрицу объектов-признаков
\begin{equation}
	\label{eq10}
	\bX = [\bx_2, \ldots, \bx_N]\T = [x^i_j] \in \mathbb{R}^{(N-1) \times d}
\end{equation}
и вектор, компонентами которого являются разности значений одного и того же вокселя в разных снимках,
\begin{equation}
	\label{eq11}
	\mathbf{\Delta}_{ijk} = [\delta^2_{ijk}, \ldots, \delta^N_{ijk}]\T \in \mathbb{R}^{N-1}.
\end{equation}

Решение можно записать в виде
\begin{equation}
	\label{eq12}
	\hat{\bw}_{ijk} = (\bX\T \bX + \alpha \mathbf{I})^{-1} \bX\T \mathbf{\Delta}_{ijk}.
\end{equation}

Получим формулу для восстановленных снимков фМРТ. Введем матрицу весов
\begin{equation}
	\label{eq13}
	\hat{\bW} = [\hat{\bw}_1, \ldots, \hat{\bw}_{XYZ}]\T = [\hat{w}^i_j] \in \mathbb{R}^{XYZ \times d}.
\end{equation}

Введем для тензоров $\bs_{\ell}, \bdelta_{\ell} \in \mathbb{R}^{X \times Y \times Z}$ векторы
\[ \bs_{\ell}^{R} = [ v^{\ell}_1, \ldots, v^{\ell}_{XYZ} ]\T,\
	\bdelta_{\ell}^{R} = [ \delta^{\ell}_1, \ldots, \delta^{\ell}_{XYZ} ]\T \in \mathbb{R}^{XYZ}. \]

Тогда вектор восстановленного снимка находится по формуле
\begin{equation}
	\label{eq14}
	\hat{\bs}_{\ell}^{R} = \bs_{\ell-1}^{R} + \hat{\bdelta}_{\ell}^{R} = \bs_{\ell-1}^{R} + \hat{\mathbf{W}} \mathbf{x}_{\ell}.
\end{equation}

\section{Вычислительный эксперимент}

Для анализа работоспособности предложенного метода и проверки гипотез
проведен вычислительный эксперимент.

В качестве данных использовалась выборка, представленная в работе \citep{Berezutskaya2022}.
Набор данных содержит результаты обследования 63 испытуемых.
Для тридцати из них известны показания фМРТ.
Среди них 16 мужчин и 14 женщин в возрасте от 7 до 47 лет.
Средний возраст испытуемого~--- 22 года.

Характеристики выборки: продолжительность обследования,
частота кадров видеоряда и снимков фМРТ, размерности изображений
и снимков приведены в Таблице~\ref{table:sample}

\begin{table}[h!]
	\centering
	\caption{Описание выборки}
	\begin{tabular}{|c|c|c|}
		\hline
		Название                       & Обозначение & Значение             \\
		\hline \hline
		Продолжительность обследования & $t$         & 390 с                \\ \hline
		Частота кадров видео           & $\nu$       & 25 $\text{с}^{-1}$   \\ \hline
		Частота снимков фМРТ           & $\mu$       & 1,64 $\text{с}^{-1}$ \\ \hline
		Размерности изображения        & $W, H, C$   & 640, 480, 3          \\ \hline
		Размерности снимка             & $X, Y, Z$   & 40, 64, 64           \\ \hline
	\end{tabular}
	\label{table:sample}
\end{table}

Произведено разделение выборки на тренировочную и тестовую в соотношении 70\% и 30\% соответственно.
Критерием качества восстановления снимка фМРТ служит MSE~--- сумма квадратов отклонений
между истинным и восстановленным снимками, усредненная по всем вокселям каждого снимка
из тестовой выборки.

Для сокращения времени работы алгоритма производится предварительное сжатие снимка фМРТ
с помощью сверточного слоя MaxPool3D. Рассматриваются коэффициенты сжатия 1, 2, 4 и 8.
Значения вокселей нормализуются на $[0; 1]$ процедурой MinMaxScale.

\paragraph*{Демонстрация работы метода.}

На Рис.~\ref*{fig:example} представлены срезы истинного и восстановленного снимков из
тестовой выборки. На Рис.\myfigref{fig:example}{fig:example-c} можно наблюдать разность между ними.
Для демонстрации работы алгоритма был выбран 4-ый испытуемый, $\Delta t = 5 \text{ с}$, коэффициент сжатия 1, коэффициент регуляризации
$\alpha = 1000$. Рассмотрен 20-ый срез по первой координате 100-го снимка в последовательности.
Так как значения вокселей нормированы на отрезок $[0; 1]$, то ошибка порядка $10^{-3} \div 10^{-2}$
свидетельствует о достаточно точном предсказании.

\begin{figure}[h!]
	\centering
	\subfloat[Истинный]{\label{fig:example-a}{\includegraphics[width=0.33\textwidth]{sub-04-5-1-1000-100-20-_-_-test.png}}}
	\hfill
	\subfloat[Восстановленный]{\label{fig:example-b}{\includegraphics[width=0.33\textwidth]{sub-04-5-1-1000-100-20-_-_-predicted.png}}}
	\hfill
	\subfloat[Разность]{\label{fig:example-c}{\includegraphics[width=0.33\textwidth]{sub-04-5-1-1000-100-20-_-_-difference.png}}}
	\caption{Срез снимка фМРТ из тестовой выборки}
	\label{fig:example}
\end{figure}

\paragraph*{Анализ времени задержки.}

Исследована зависимость качества восстановления от времени задержки.
Для примера был выбран 47-ой испытуемый.
На левом графике Рис.~\ref{fig:mse-dt} представлена зависимость метрики MSE
от времени задержки $\Delta t$.
Исследование \citep{anderson2006} подтверждает, что наиболее активная часть мозга
при рассматриваемом обследовании~--- затылочная доля.
Остальные части вносят шум в рассматриваемую зависимость.
В настоящей работе проведена локализация вышеупомянутой области, 
что можно наблюдать на Рис.~\ref{fig:local}.
Для локализации области отсекаются нижняя треть и правые две трети объемного
томографического изображения.
Красным цветом выделена та зона, в которую попадают 3\% наиболее 
изменяющихся вокселей затылочной доли.
Для этого все воксели локализованной области были упорядочены по 
убыванию суммарного абсолютного изменения значений.
Далее были выбраны 3\% вокселей с наибольшими изменениями.
Произведен пересчет метрики MSE именно на этой части снимка.
Соответствующий график приведен справа на Рис.~\ref{fig:mse-dt}.

\begin{figure}[h!]
	\centering
	\subfloat[Истинный]{\label{fig:local-a}{\includegraphics[width=0.33\textwidth]{local/sub-47-5-1-1000-37-20-_-_-test.png}}}
	\hfill
	\subfloat[Восстановленный]{\label{fig:local-b}{\includegraphics[width=0.33\textwidth]{local/sub-47-5-1-1000-37-20-_-_-predicted.png}}}
	\hfill
	\subfloat[Разность]{\label{fig:local-c}{\includegraphics[width=0.33\textwidth]{local/sub-47-5-1-1000-37-20-_-_-difference.png}}}
	\caption{Локализация наиболее активной зоны}
	\label{fig:local}
\end{figure}

\begin{figure}[h!]
	\centering
	\includegraphics[width=\textwidth]{mse_dt.pdf}
	\caption{Зависимость метрики MSE от времени задержки}
	\label{fig:mse-dt}
\end{figure}

\paragraph*{Подбор оптимального коэффициента регуляризации.}

Проведен анализ зависимости MSE от коэффициента регуляризации $\alpha$.
Рассматривались коэффициенты сжатия 1, 2, 4 и 8.
Соответствующие графики приведены на Рис.~\ref{fig:mse-alpha}.
Для построения графика производилось усреднение по испытуемым.
Обозначены границы среднеквадратичного отклонения.
Из графиков видно, что оптимальное значение коэффициента $\alpha \approx 1000$.

\begin{figure}[h!]
	\centering
	\includegraphics[width=0.65\textwidth]{subs_MSE_alpha.pdf}
	\caption{Зависимость метрики MSE от коэффициента регуляризации $\alpha$ на снимках из тестовой выборки}
	\label{fig:mse-alpha}
\end{figure}

\paragraph*{Анализ распределения весов модели.}

Построен график распределения значений компонент вектора весов модели.
Для построения производилось усреднение по всем вокселям для 4-го испытуемого.
Результат представлен на Рис.~\ref{fig:w-distr}.
Веса модели не лежат в окрестности какого-то конкретного значения, 
то есть их распределение не вырождено.
Этот результат вполне согласуется с реальностью, поскольку человек во время просмотра
обращает внимание на определенные части кадра, например, персонажей или
другие детали.

\begin{figure}[h!]
	\centering
	\includegraphics[width=0.65\textwidth]{distribution.pdf}
	\caption{Распределение значений компонент вектора весов}
	\label{fig:w-distr}
\end{figure}

\paragraph*{Гипотеза инвариантности весов модели относительно человека.}

Проведена проверка гипотезы инвариантности весов модели относительно человека:
можно ли проводить восстановление снимка фМРТ одного испытуемого, используя
матрицу весов другого. Использовалась метрика MSE на тестовой выборке.
Результаты представлены в Таблице~\ref{table:inv}.
Рассмотрены 4-ый и 7-ый испытуемые. Матрица весов 4-го использовалась для восстановления
снимков 7-го.
Значения MSE практически совпадают. 

\begin{table}[h!]
	\centering
	\caption{Проверка гипотезы инвариантности весов модели относительно человека}
	\begin{tabular}{|c|c|c|}
		\hline
		Матрица весов & Истинная             & Подмешанная          \\ \hline \hline
		MSE           & $1.02 \cdot 10^{-4}$ & $1.05 \cdot 10^{-4}$ \\ \hline
	\end{tabular}
	\label{table:inv}
\end{table}

Аналогичный эксперимент проведен для каждой пары испытуемых.
Полученные результаты можно наблюдать на Рис.~\ref{fig:heatmap},
который был получен следующим образом.
Рассматривается некоторый испытуемый (соответствует строке матрицы), 
для него вычисляется MSE~--- <<истинный>>.
Далее рассматривается другой испытуемый (соответствует столбцу матрицы),
берется его матрица весов, и с помощью нее делается предсказание для первого 
испытуемого, затем вычисляется MSE~--- <<подмешанный>>. 
Разность между полученными MSE в процентах от <<истинного>> заносится в матрицу.
Положительное значение означает, что <<подмешанный>> MSE больше, чем <<истинный>>.
Отрицательное~--- что <<подмешанный>> меньше.
Идеальная модель должна приводить только к положительным значениям отклонений, однако,
как видно из Рис.~\ref{fig:heatmap}, в матрице есть и отрицательные значения.
Тем не менее, они достаточно малы, а именно соответствуют отклонениям порядка 1\%.
Это можно объяснить тем, что модель достаточно простая и подвержена переобучению.
Однако это не мешает сделать вывод о том, что данные не противоречат гипотезе 
об инвариантности весов модели относительно человека.

\begin{figure}[h!]
	\centering
	\includegraphics[width=0.5\textwidth]{heatmap.pdf}
	\caption{Изменение MSE при предсказании на подмешанной матрице весов (в процентах)}
	\label{fig:heatmap}
\end{figure}

\paragraph*{Обобщающая способность метода.}

Рассмотрено качество работы метода на случайно сгенерированных данных. 
В качестве матрицы объекты-признаки $\bX$ взята матрица 
случайных чисел из равномерного распределения на $[0; 1]$.
Произведено сравнение с результатами на настоящей матрице признакового описания.
К первому снимку 35-го испытуемого последовательно прибавляются все восстановленные
изменения значений вокселей.
В результате имеем последний снимок последовательности. На Рис.~\ref*{fig:recover}
представлены срезы последнего истинного и восстановленного снимков из тестовой выборки.
На Рис.\myfigref{fig:recover}{fig:recover-c} можно наблюдать разность между ними.
Результаты на случайных данных продемонстрированы на Рис.~\ref*{fig:random}.
Можно видеть, что разность между истинным и восстановленным снимками при работе со случайными данными
значительно выше, что подтверждает наличие корреляции между показаниями датчиков и
изображениями из видеоряда. Численные результаты приведены в Таблице~\ref{table:random}.

\begin{figure}[h!]
	\centering
	\subfloat[Истинный]{\label{fig:recover-a}{\includegraphics[width=0.33\textwidth]{original/sub-35-5-1-1000--1-20-_-_-recovered-test.png}}}
	\hfill
	\subfloat[Восстановленный]{\label{fig:recover-b}{\includegraphics[width=0.33\textwidth]{original/sub-35-5-1-1000--1-20-_-_-recovered-predicted.png}}}
	\hfill
	\subfloat[Разность]{\label{fig:recover-c}{\includegraphics[width=0.33\textwidth]{original/sub-35-5-1-1000--1-20-_-_-recovered-difference.png}}}
	\caption{Срез снимка фМРТ из тестовой выборки}
	\label{fig:recover}
\end{figure}

\begin{figure}[h!]
	\centering
	\subfloat[Истинный]{\label{fig:random-a}{\includegraphics[width=0.33\textwidth]{noised/sub-35-5-1-1000--1-20-_-_-recovered-test.png}}}
	\hfill
	\subfloat[Восстановленный]{\label{fig:random-b}{\includegraphics[width=0.33\textwidth]{noised/sub-35-5-1-1000--1-20-_-_-recovered-predicted.png}}}
	\hfill
	\subfloat[Разность]{\label{fig:random-c}{\includegraphics[width=0.33\textwidth]{noised/sub-35-5-1-1000--1-20-_-_-recovered-difference.png}}}
	\caption{Срез снимка фМРТ из тестовой выборки (на случайных данных)}
	\label{fig:random}
\end{figure}

\begin{table}[h!]
	\centering
	\caption{Качество работы метода на случайных данных}
	\begin{tabular}{|c|c|c|}
		\hline
		Выборка & Истинная          & Случайные данные \\ \hline \hline
		MSE     & $2 \cdot 10^{-3}$ & $10^{-1}$        \\ \hline
	\end{tabular}
	\label{table:random}
\end{table}

\newpage

\section{Заключение}

В работе рассматривалась задача восстановления зависимости между показаниями
датчиков фМРТ и восприятием внешнего мира человеком.
Был предложен метод аппроксимации последовательности снимков фМРТ по видеоряду,
просматриваемому человеком.
В ходе экспериментов была показана справедливость гипотезы о линейной зависимости между данными.
Кроме того, была подтверждена гипотеза о взаимосвязи снимков в последовательности.
Проверена гипотеза инвариантности весов модели относительно человека.

\newpage

\bibliographystyle{unsrt}
\bibliography{fMRI_2023.bib}

\end{document}