\documentclass[10pt]{beamer}
\usepackage[utf8]{inputenc}
\usepackage[english,russian]{babel}
\usepackage{amsmath,mathrsfs,mathtext}
\usepackage{graphicx,epsfig}
\usepackage{caption}
\usepackage{subfig}
\usepackage{bm}
\usetheme{Warsaw}%{Singapore}%{Warsaw}%{Warsaw}%{Darmstadt}
\usecolortheme{sidebartab}
%\definecolor{beamer@blendedblue}{RGB}{15,120,80}

%% отключение навигации по слайдам справа внизу
\beamertemplatenavigationsymbolsempty

%% отступы слева и справа (как бы поля)
\setbeamersize{text margin left=1.5em,text margin right=1.5em} 

\newcommand{\bz}{\mathbf{z}}
\newcommand{\bx}{\mathbf{x}}
\newcommand{\by}{\mathbf{y}}
\newcommand{\bv}{\mathbf{v}}
\newcommand{\bw}{\mathbf{w}}
\newcommand{\ba}{\mathbf{a}}
\newcommand{\bb}{\mathbf{b}}
\newcommand{\bp}{\mathbf{p}}
\newcommand{\bq}{\mathbf{q}}
\newcommand{\bt}{\mathbf{t}}
\newcommand{\bu}{\mathbf{u}}
\newcommand{\bs}{\mathbf{s}}
\newcommand{\bT}{\mathbf{T}}
\newcommand{\bX}{\mathbf{X}}
\newcommand{\bZ}{\mathbf{Z}}
\newcommand{\bS}{\mathbf{S}}
\newcommand{\bH}{\mathbf{H}}
\newcommand{\bW}{\mathbf{W}}
\newcommand{\bY}{\mathbf{Y}}
\newcommand{\bU}{\mathbf{U}}
\newcommand{\bQ}{\mathbf{Q}}
\newcommand{\bP}{\mathbf{P}}
\newcommand{\bA}{\mathbf{A}}
\newcommand{\bB}{\mathbf{B}}
\newcommand{\bC}{\mathbf{C}}
\newcommand{\bE}{\mathbf{E}}
\newcommand{\bF}{\mathbf{F}}
\newcommand{\bomega}{\boldsymbol{\omega}}
\newcommand{\btheta}{\boldsymbol{\theta}}
\newcommand{\bgamma}{\boldsymbol{\gamma}}
\newcommand{\bdelta}{\boldsymbol{\delta}}
\newcommand{\bPsi}{\boldsymbol{\Psi}}
\newcommand{\bpsi}{\boldsymbol{\psi}}
\newcommand{\bxi}{\boldsymbol{\xi}}
\newcommand{\bchi}{\boldsymbol{\chi}}
\newcommand{\bzeta}{\boldsymbol{\zeta}}
\newcommand{\blambda}{\boldsymbol{\lambda}}
\newcommand{\beps}{\boldsymbol{\varepsilon}}
\newcommand{\bZeta}{\boldsymbol{Z}}
% mathcal
\newcommand{\cX}{\mathcal{X}}
\newcommand{\cY}{\mathcal{Y}}
\newcommand{\cW}{\mathcal{W}}

\newcommand{\dH}{\mathds{H}}
\newcommand{\dR}{\mathds{R}}
% transpose
\newcommand{\T}{^{\mathsf{T}}}

% \renewcommand{\shorttitle}{\textit{arXiv} Шаблон}
\renewcommand{\epsilon}{\ensuremath{\varepsilon}}
\renewcommand{\phi}{\ensuremath{\varphi}}
\renewcommand{\kappa}{\ensuremath{\varkappa}}
\renewcommand{\le}{\ensuremath{\leqslant}}
\renewcommand{\leq}{\ensuremath{\leqslant}}
\renewcommand{\ge}{\ensuremath{\geqslant}}
\renewcommand{\geq}{\ensuremath{\geqslant}}
\renewcommand{\emptyset}{\varnothing}

\DeclareMathOperator*{\argmax}{arg\,max}  % in your preamble
\DeclareMathOperator*{\argmin}{arg\,min}  % in your preamble 

\graphicspath{{../figures}, {../figures_delta}}

%\usepackage{biblatex}
\usepackage{hyperref}
\renewcommand{\thefootnote}{\arabic{footnote}}

%----------------------------------------------------------------------------------------------------------
\title[\hbox to 56mm{Восстановление снимков фМРТ \hfill\insertframenumber\,/\,\inserttotalframenumber}]
{Восстановление снимков фМРТ \\ по просматриваемому видеоряду}
\author[Н.\,С.~Киселев]{\large Никита Сергеевич Киселев}
\institute{\large
Московский физико-технический институт\par
(национальный исследовательский университет)}

\date{\footnotesize{\emph{Курс:} Автоматизация научных исследований\par (Моя первая научная статья)/Группа 003, весна 2023 \\
\par\emph{Эксперт:} А.\,В.~Грабовой
}}
%----------------------------------------------------------------------------------------------------------
\begin{document}
%----------------------------------------------------------------------------------------------------------
\begin{frame}
%\thispagestyle{empty}
\titlepage
\end{frame}
%-----------------------------------------------------------------------------------------------------
\begin{frame}{Цель исследования}
    \begin{block}{Проблема}
        Восстановление показаний датчиков фМРТ при взаимодействии человека с внешним миром.
    \end{block}
    \begin{block}{Цель}
        Аппроксимация последовательности снимков фМРТ по видеоряду,
	    просматриваемому человеком.
    \end{block}
    \begin{block}{Предлагается}
        \begin{enumerate}
            \item Восстановление изменения снимка фМРТ с учетом времени задержки~$\Delta t$.
            \item Исследование свойств построенного метода и проверка гипотез.
        \end{enumerate}
    \end{block}
\end{frame}
%----------------------------------------------------------------------------------------------------------
\begin{frame}{Постановка задачи}
    \begin{figure}[h!]
        \centering
        \includegraphics[width=0.9\textwidth]{scheme.png}
    \end{figure}
\end{frame}
%----------------------------------------------------------------------------------------------------------
\begin{frame}{Постановка задачи}
    Пусть задана частота кадров $\nu \in \mathbb{R}$ и продолжительность $t \in \mathbb{R}$ видеоряда. 
	Задан видеоряд
	\begin{equation*}
		\label{eq1}
		\mathbf{P} = [\mathbf{p}_1, \ldots, \mathbf{p}_{\nu t}], \quad\
		\mathbf{p}_{\ell} \in \mathbb{R}^{W \times H \times C}.
	\end{equation*}

	Обозначим частоту снимков фМРТ $\mu \in \mathbb{R}$. Задана последовательность снимков 
	\begin{equation*}
		\label{eq2}
		\mathbf{S} = [\mathbf{s}_1, \ldots, \mathbf{s}_{\mu t}], \quad\
		\mathbf{s}_{\ell} \in \mathbb{R}^{X \times Y \times Z}.
	\end{equation*}

	Необходимо построить отображение, которое бы учитывало задержку $\Delta t$ между
	снимком фМРТ и видеорядом, а также предыдущие томографические показания.
    \begin{block}{}
        \begin{equation*}
            \label{eq3}
            \mathbf{g}(\mathbf{p}_1, \ldots, \mathbf{p}_{k_{\ell} - \nu \Delta t}; \mathbf{s}_1, \ldots, \mathbf{s}_{\ell-1}) = \mathbf{s}_{\ell},
        \end{equation*}
        \begin{equation*}
            \ell = 1, \ldots, \mu t, \qquad k_{\ell} = \dfrac{\ell \cdot \nu}{\mu}.
        \end{equation*}
    \end{block}
\end{frame}
%----------------------------------------------------------------------------------------------------------
\begin{frame}{Описание метода}
    \textbf{Предположение марковости:}\\
    Снимок $\mathbf{s}_{\ell}$ зависит только от изображения $\mathbf{p}_{k_\ell - \nu \Delta t}$
    и снимка $\mathbf{s}_{\ell-1}$, тогда
	\begin{equation*}
		\label{eq4}
		\mathbf{g}(\mathbf{p}_{k_{\ell} - \nu \Delta t}) = \mathbf{s}_{\ell} - \mathbf{s}_{\ell-1} = \bm{\delta}_{\ell}, \ \ell = 2, \ldots, \mu t.
	\end{equation*}

    Отображение $\mathbf{g}: \mathbf{P} \to \mathbf{S}$ представимо в виде композиции
    \[ \mathbf{g} = \bm{\varphi} \circ \bm{\psi}, \]
    \vspace{-0.8cm}
    \begin{align*}
        &\bm{\psi}: \mathbf{P} \to \mathbb{R}^d
        \text{~--- векторизация изображения,}\\
        &\bm{\varphi}: \mathbb{R}^d \to \mathbf{S}
        \text{~--- восстанавливаемое отображение.}
    \end{align*}

    Для векторизации используется ResNet152 без последнего линейного слоя.
    Суммарное число пар (изображение, снимок): $N = \mu (t - \Delta t)$.\\
    Таким образом, для каждого вокселя задана выборка
    \[ \mathfrak{D}_{ijk} = \{(\mathbf{x}_{\ell}, \delta^{\ell}_{ijk}) \ | \ {\ell} = 2, \ldots, N \}. \]
    \vspace{-0.5cm}
    \begin{itemize}
        \item $\mathbf{x}_{\ell} = [x^{\ell}_1, \ldots, x^{\ell}_{d}]\T \in \mathbb{R}^{d}$~--- признаки изображения;
        \item $\bm{\delta}_{\ell} = [\delta^{\ell}_{ijk}] \in \mathbb{R}^{X \times Y \times Z}$~--- разность между двумя последовательными снимками фМРТ.
    \end{itemize}

\end{frame}
%----------------------------------------------------------------------------------------------------------
\begin{frame}{Модель, функция потерь и решение}
    Используется линейная модель с вектором параметров 
	\[ \mathbf{w}_{ijk} = [w^{ijk}_1, \ldots, w^{ijk}_{d}]\T \in \mathbb{R}^{d}: \]
	\begin{equation*}
		\label{eq5}
		f_{ijk}(\mathbf{x}, \mathbf{w}_{ijk}) = \langle \mathbf{x}, \mathbf{w}_{ijk} \rangle.
	\end{equation*}
    Квадратичная функция потерь с коэффициентом $L_2$-регуляризации $\alpha \in \mathbb{R}$:
	\begin{equation*}
		\label{eq6}
		\mathcal{L}_{ijk}(\mathbf{w}_{ijk}, \Delta t) = \sum\limits_{\ell = 2}^{N} \big(f_{ijk}(\mathbf{x}_{\ell}, \mathbf{w}_{ijk}) - \delta^{\ell}_{ijk}\big)^2 + \alpha \| \mathbf{w}_{ijk} \|_2^2,
	\end{equation*}
    Требуется найти параметры, доставляющие минимум функционалу потерь при фиксированном $\Delta t$:
    \begin{equation*}
        \label{eq7}
        \hat{\mathbf{w}}_{ijk} = \argmin_{\mathbf{w}_{ijk}} \mathcal{L}_{ijk}(\mathbf{w}_{ijk}, \Delta t).
    \end{equation*}
    Решение для каждого вокселя находится методом наименьших квадратов:
    \begin{equation*}
		\label{eq8}
		\hat{\bw}_{ijk} = (\bX\T \bX + \alpha \mathbf{I})^{-1} \bX\T \mathbf{\Delta v}_{ijk}.
	\end{equation*}
    \vspace{-0.5cm}
    \begin{itemize}
        \item $\bX = [\bx_2\T, \ldots, \bx_N\T]\T = [x^i_j] \in \mathbb{R}^{(N-1) \times d}$~--- матрица объекты-признаки;
        \item $\mathbf{\Delta v}_{ijk} = [\delta^2_{ijk}, \ldots, \delta^N_{ijk}]\T \in \mathbb{R}^{N-1}$~--- вектор изменения вокселя в снимках.
    \end{itemize}
\end{frame}
%----------------------------------------------------------------------------------------------------------
\begin{frame}{Вычислительный эксперимент}
    \begin{block}{Цель}
        \begin{enumerate}
            \item Проверка работоспособности предложенного метода.
            \item Исследование зависимости качества восстановления от гиперпараметра $\Delta t$.
            \item Проверка гипотез:
                    \begin{itemize}
                        \item линейная зависимость между данными;
                        \item взаимосвязь снимков в последовательности;
                        \item инвариантность весов модели относительно человека.
                    \end{itemize}
        \end{enumerate}
    \end{block}
    \vspace{-0.2cm}
    \begin{block}{Данные}
        Реальное фМРТ-обследование\footnotemark 30 испытуемых разного пола и возраста.
        Каждый из них просматривал короткий аудиовизуальный фильм.
        Продолжительность фильма $t = 390$~с, частота кадров $\nu = 25$~$\text{с}^{-1}$.\\
        Частота снимков фМРТ $\mu = 1.64$~$\text{с}^{-1}$.
    \end{block}
    \vspace{-0.1cm}
    Производится предварительное сжатие снимка фМРТ с помощью сверточного слоя MaxPool3D.
	Значения вокселей нормализуются на $[0; 1]$.
    \footnotetext[1]{
    Julia Berezutskaya et al. (2022). Open multimodal iEEG-fMRI dataset from naturalistic stimulation
    with a short audiovisual film.    
    \href{https://doi.org/10.1038/s41597-022-01173-0}{https://doi.org/10.1038/s41597-022-01173-0}
    }
\end{frame}
%----------------------------------------------------------------------------------------------------------
\begin{frame}{Зависимость от гиперпараметра $\Delta t$}
    Зависимость метрики MSE от гиперпараметра $\Delta t$.
    Использовалось предварительное 8-кратное сжатие снимка.
    Производилось усреднение по испытуемым.
	Обозначены границы среднеквадратичного отклонения.
    \begin{figure}[h!]
		\centering
		\includegraphics[width=0.65\textwidth]{subs_delta_MSE_dt.pdf}
		\label{fig:3}
	\end{figure}
    \vspace{-0.2cm}
    Качество восстановления зависит от значения $\Delta t$. 
    Наблюдается характерный минимум при $\Delta t \approx 10 \text{ с}$.
\end{frame}
%----------------------------------------------------------------------------------------------------------
\begin{frame}{Восстановленный снимок}
    Срезы истинного и восстановленного снимков из тестовой выборки.
    Можно наблюдать разность между ними.
    \begin{figure}[h!]
		\centering
		\subfloat[Истинный]{\label{fig:4a}{\includegraphics[width=0.33\textwidth]{sub-04-5-1-1000/sub-04-5-1-1000-100-20-_-_-test.png}}}
		\hfill
		\subfloat[Восстановленный]{\label{fig:4b}{\includegraphics[width=0.33\textwidth]{sub-04-5-1-1000/sub-04-5-1-1000-100-20-_-_-predicted.png}}}
		\hfill
		\subfloat[Разность]{\label{fig:4c}{\includegraphics[width=0.33\textwidth]{sub-04-5-1-1000/sub-04-5-1-1000-100-20-_-_-difference.png}}}
		\label{fig:4}
	\end{figure}
    Усредненная по всем снимкам и вокселям $\text{MSE} = 1.09 \cdot 10^{-4}$.
    Значения вокселей лежат в отрезке $[0; 1]$, поэтому ошибка на одном снимке
    порядка $10^{-3} \div 10^{-2}$ свидетельствует о достаточно точном предсказании.
\end{frame}
%----------------------------------------------------------------------------------------------------------
\begin{frame}{Зависимость от коэффициента регуляризации}
    Зависимость метрики MSE от коэффициента регуляризации $\alpha$.
    Рассматривались коэффициенты сжатия 1, 2, 4 и 8.
    Производилось усреднение по испытуемым.
	Обозначены границы среднеквадратичного отклонения.
    \begin{figure}[h!]
		\centering
		\includegraphics[width=0.65\textwidth]{subs_MSE_alpha.pdf}
		\label{fig:5}
	\end{figure}
    \vspace{-0.2cm}
    Оптимальное значение коэффициента $\alpha \approx 100$.
\end{frame}
%----------------------------------------------------------------------------------------------------------
\begin{frame}{Распределение весов в среднем по всем вокселям}
    График распределения значений компонент вектора весов модели.
    Производилось усреднение по всем вокселям фиксированного снимка.
    \begin{figure}[h!]
		\centering
		\includegraphics[width=0.65\textwidth]{mean_weight_distribution.pdf}
		\label{fig:6}
	\end{figure}
    Веса модели распределены нормально, а не лежат в окрестности какого-то определенного
    значения.
\end{frame}
%----------------------------------------------------------------------------------------------------------
\begin{frame}{Инвариантность весов относительно человека}
    \textbf{Предположение:} зависимость между видеорядом и снимками фМРТ у разных испытуемых
    имеет схожий характер, а время задержки $\Delta t$ практически совпадает.

    Проведена проверка гипотезы инвариантности весов модели относительно человека:
	можно ли восстановить снимок фМРТ одного испытуемого, используя
	матрицу весов другого. Использовалась метрика MSE на тестовой выборке.
    \begin{table}[h!]
		\centering
		\begin{tabular}{|c|c|c|}
			\hline
			Матрица весов	&	Истинная	&	Подмешанная\footnotemark[1] \\ \hline \hline
			MSE		& 	$1.02 \cdot 10^{-4}$	 &		$1.05 \cdot 10^{-4}$ \\ \hline
		\end{tabular}
		\label{table:1}
	\end{table}
    Значения MSE практически совпадают. Это свидетельствует о справедливости рассматриваемой гипотезы. 
    \footnotetext[1]{Предсказание снимков одного человека с использованием весов модели другого}
\end{frame}
%----------------------------------------------------------------------------------------------------------
\begin{frame}{Проверка работы на случайном шуме}
    Рассмотрено качество работы метода на случайном шуме. В качестве матрицы объекты-признаки
	взята матрица случайных чисел $\mathbf{X} = [x^i_j], \ x^i_j \sim U[0, 1]$. 
    Ниже приведены срезы последнего снимка, восстановленного
    последовательно по всем предсказанным изменениям, и значения метрики MSE.
    \vspace{-0.5cm}
	\begin{figure}[h!]
		\centering
		\subfloat[Истинный]{\label{fig:8a}{\includegraphics[width=0.33\textwidth]{sub-35-5-1-1000/noised/sub-35-5-1-1000--1-20-_-_-recovered-test.png}}}
		\hfill
		\subfloat[Восстановленный]{\label{fig:8b}{\includegraphics[width=0.33\textwidth]{sub-35-5-1-1000/noised/sub-35-5-1-1000--1-20-_-_-recovered-predicted.png}}}
		\hfill
		\subfloat[Разность]{\label{fig:8c}{\includegraphics[width=0.33\textwidth]{sub-35-5-1-1000/noised/sub-35-5-1-1000--1-20-_-_-recovered-difference.png}}}
		\label{fig:8}
	\end{figure}
    \vspace{-0.2cm}
    \begin{table}[h!]
		\centering
		\begin{tabular}{|c|c|c|}
			\hline
			Выборка	&	Истинная	&	Случайный шум \\ \hline \hline
			MSE		& 	$2 \cdot 10^{-3}$	 &		$10^{-1}$ \\ \hline
		\end{tabular}
		\label{table:2}
	\end{table}
    Большое значение MSE при работе на случайном шуме свидетельствует о наличии зависимости 
    между видеорядом и снимками фМРТ, которую улавливает предложенная модель. 
\end{frame}
%----------------------------------------------------------------------------------------------------------
\begin{frame}{Заключение}
    \begin{enumerate}
        \item Предложен метод аппроксимации последовательности снимков фМРТ по видеоряду,
              просматриваемому человеком.
        \item Проверены гипотезы:
            \begin{itemize}
                \item о линейной зависимости между видеорядом и снимками фМРТ;
                \item о взаимосвязи снимков в последовательности;
                \item об инвариантности весов модели относительно человека.
            \end{itemize}
    \end{enumerate}
\end{frame}
%----------------------------------------------------------------------------------------------------------
\end{document} 